Vim�UnDo��$e?J�f
��Ϥކ�6��o
:E^�R*Vlh�\?j�_�����\?j���\documentclass{svproc}\usepackage{times}\usepackage{tabularx}\usepackage{multicol}t\usepackage[bookmarks,bookmarksnumbered,colorlinks, urlcolor={black}, linkcolor={black},citecolor={black}]{hyperref}\usepackage{mathtools}\usepackage{xspace}%\usepackage{amsmath,amssymb,mathrsfs}\usepackage{thmtools}\usepackage{thm-restate}\usepackage{graphicx} \usepackage{todonotes} \usepackage{tikz}\usepackage{proof}\usepackage{etoolbox}\usepackage{textcomp}\usepackage{comment}\usepackage{enumerate}   ,\usepackage[ruled,vlined,noend]{algorithm2e}\usepackage{color}\usepackage{subcaption}\usepackage{bm}\usepackage{sidecap}\SetCommentSty{mycommfont}\usepackage{url}\def\UrlFont{\rmfamily}\input{defs.tex}\newcommand\blfootnote[1]{%
  \begingroup+  \renewcommand\thefootnote{}\footnote{#1}%  \addtocounter{footnote}{-1}%  \endgroup}\begin{document}\title{GFinding plans subject to stipulations on what information they divulge%}Q\author{Yulin Zhang\inst{1}, Dylan A. Shell\inst{1} \and Jason M. O'Kane\inst{2}}\institute{$%\email{yulinzhang, dshell@tamu.edu}0Texas A\&M University, College Station TX, USA\\\and %\email{jokane@cse.sc.edu}/University of South Carolina, Columbia SC, USA%}P%DAS TODO: - Mention the importance of worst-case reasoning (and maybe set-based4%            uncertainty?) for adversarial settings.a%          - Be explicit about this being filtering, strictly sequential processing of the stream
\maketitleP%%%%%%%%%%%%%%%%%%%%%%%%%%%%%%%%%%%%%%%%%%%%%%%%%%%%%%%%%%%%%%%%%%%%%%%%%%%%%%%%%\vspace*{-20pt}\begin{abstract}GMotivated by applications where privacy is important, we study planningNproblems for robots acting in the presence of an observer.  We first formulateKand then solve planning problems subject to stipulations on the informationNdivulged during plan execution---the appropriate solution concept being both aOplan and an information disclosure policy.  We pose this class of problem underNa worst-case model within the framework of procrustean graphs, formulating theHdisclosure policy as a particular type of map on edge labels.  We deviseJalgorithms that, given a planning problem supplemented with an informationOstipulation, can find a plan, associated disclosure policy, or both jointly, if,and only if some exists. The pair together, 6comprising the plan and associated disclosure policy, may depend subtly on'additional information available to theOobserver, such as whether the observer knows the robot's plan (e.g., leaked viaa side-channel).  l%This additional information may be the same as the plan to be executed or may %differ. For all these cases,AOur implementation finds a plan and a suitable disclosure policy,Gjointly, when any such pair exists, albeit for small problem instances.\end{abstract}%\vspace*{-40pt}2\blfootnote{This work was supported by NSF awards H\href{http://nsf.gov/awardsearch/showAward?AWD_ID=1453652}{IIS-1453652},H\href{http://nsf.gov/awardsearch/showAward?AWD_ID=1527436}{IIS-1527436},and I\href{http://nsf.gov/awardsearch/showAward?AWD_ID=1526862}{IIS-1526862}.}\vspace*{-26pt}\section{Introduction}\vspace*{-6pt}LIn 2017, iRobot announced that they intended to sell maps of people's homes,Eas generated by their robot vacuum cleaners.  The result was a publicPoutcry~\cite{privacynews}.  It is increasingly clear that, as robots become partFof our everyday lives, the information they could collect (indeed, mayG\emph{need} to collect to function) can be both sensitive and valuable.KInformation about a robot's internal state and its estimates of the world'sGstate are leaked by status displays, logged data, actions executed, andPinformation outputted --- often what the robot is tasked with doing. The tension+between utility and privacy is fundamental.M%information outputted. Often producing this information is what the robot isL%tasked with doing, underscoring the fundamental tension between utility and	%privacy.1Typically, robots strive to decrease uncertainty.0%Robots typically strive to decrease uncertaintyE%to ensure safe operation, to narrow the selection of future actions,1%or to gain information as a direct objective.   NSome prior work, albeit limited, has illustrated how to cultivate uncertainty,Nexamining how to constrain a robot's beliefs so that it never learns sensitiveHinformation (cf.~\cite{OKa08,ddf2015,zhang18complete}). In so doing, oneIprecludes sensitive information being disclosed to any adversary. But not:disclosing secrets by simply never knowing any, limits theKapplicability of the approach severely. This paper proposes a more general,Awider-reaching model for privacy, beyond mere ing\'{e}nue robots.JThis article posits a potentially adversarial observer and then stipulates"%specifies, or we say stipulates, Lproperties of what shall be divulged.  The stipulation describes informationIthat must be communicated (being required to perform the task) as well asOinformation (confidential information potentially violating the user's privacy)Othat shouldn't be.  Practical scenarios where this model applies include: ($i$)privacy-aware care robots thatassist the housebound,5%who are mobility impaired, or in need assistance or Nproviding nursing care; ($ii$) inspection of sensitive facilities by robots toFcertify compliance with regulatory agreements, whilst protecting otherIproprietary or secret details; ($iii$) sending data remotely to computing-services on untrusted cloud infrastructure.  %\begin{figure}%\vspace*{-16pt}%\centering!%\begin{minipage}{0.40\linewidth}1%\includegraphics[scale=0.43]{figure/facility1}\\%%\small\centering Pebble bed facility%\end{minipage}%!%\begin{minipage}{0.40\linewidth}1%\includegraphics[scale=0.43]{figure/facility2}\\!%\small\centering Breeder reactor%\end{minipage}%!%\begin{minipage}{0.25\linewidth}1%\includegraphics[scale=0.30]{figure/facility3}\\%~%\end{minipage}1%\caption{{\bf Nuclear Inspection Scenario\;\;}% I%A robot performs a classification task to determine radioactivity state,V%\gobble{(orange = violation or green = admissible)} but the precise location where itW%takes the measurements depends on the type of facility.  However, the type of facility>%is sensitive information and, unless we can guarantee that itK%will not be disclosed, no inspection will be granted. \label{fig:nuclear}}%%%\vspace*{-16pt}
%\end{figure}X% <yulin> I think the figure can be made compact by using side captions, and making the % caption shorter. <\yulin>%\vspace*{-15pt}\setlength{\textfloatsep}{5pt}\begin{SCfigure}\hspace*{-5pt} \begin{minipage}{0.25\textwidth}\vspace*{-13pt}4\includegraphics[scale=0.44]{figure/facility1.pdf}\\$\small\centering Pebble bed facility\end{minipage}% \begin{minipage}{0.28\textwidth}\vspace*{-15pt}4\includegraphics[scale=0.44]{figure/facility2.pdf}\\ \small\centering Breeder reactor\end{minipage}%%\hspace*{-15pt}!%\begin{minipage}{0.15\textwidth}%\vspace*{-10pt}1%\includegraphics[scale=0.30]{figure/facility3}\\%\end{minipage}%\hspace{-10pt},\caption{{\bf Nuclear Site Inspection\;\;}% CA robot inspects a nuclear facility by taking a measurement at the 7`{\bf ?}' location, which depends on the facility type.Y%location marked with a `{\bf ?}', the specific position depending on the facility type. mBut the type of the facility is sensitive information that it must not be divulged to any external observers.\label{fig:nuclear}}%\end{SCfigure}%\vspace*{-10pt}LFigure~\ref{fig:nuclear} illustrates a scenario which, though simplistic, isOrich enough to depict several aspects of the problem.  The task requires that aNrobot determine whether some facility's processing of raw radioactive materialJmeets international treaty requirements or not.  The measurement procedureMitself depends on the type of facility as the differing physical arrangementsNof  `pebble bed' and `breeder' reactors necessitate different actions.  First,Nthe robot must actively determine the facility type (checking for the presenceNof the telltale blue light in the correct spot).  Then it can go to a locationHto make the measurement, the measurement location corresponding with theLfacility type.  But the facility type is sensitive information and the robotSmust ascertain the radioactivity state while ensuring that the facility type is not
disclosed.AWhat makes this scenario interesting is that the task is renderedVinfeasible immediately if one prescribes a policy to ensure that the robot never gainsQsensitive information.  Over and above the (classical) question of how to balanceNinformation-gathering and progress-making actions, the robot must control whatOit divulges, strategically increasing uncertainty as needed, precisely limitingIand reasoning about the `knowledge gap' between the external observer andMitself.  To solve such problems, the robot needs a carefully constructed planLand must establish a policy characterizing what information it divulges, theGformer achieving the goals set for the robot, the latter respecting allEstipulated constraints---and, of course, each depending on the other.(\subsection{Contributions and itinerary}KThis paper contributes the first formulation, to our knowledge, of planningLwhere solutions can be constrained so as to require that some information beNcommunicated and other information obscured subject to an adversarial model ofNan observer. Nor do we know of other work where both a plan and some notion ofOan interface (the disclosure policy, in our terminology) can both be solved forMjointly. The paper is organized as follows: after discussion of related work,GSection~\ref{sec:preliminary} develops the preliminaries, notation, andNformalism, Section~\ref{section:divulgedplan} addresses an important technical8detail regarding an observer's background knowledge, andLSection~\ref{section:search} finding plans that satisfy the stipulations.   HThe last section reports experiments conducted with our implementation. \section{Related work}\label{sec:related}5An important topic in HRI is expressive action (e.g.,Nsee~\cite{takayama11expressing}).  In recent years there has been a great dealJof interest in mathematical models that enable generation of communicative/plans.  Important formulations include those ofL\cite{dragan17robot,knepper17implicit}, proposing plausible models for humanKobservers (from the perspectives of presumed cost efficiency, surprisal, orKgeneralizations thereof). In this prior work, conveying information becomesOpart of an optimization objective, whereas we treat it as a constraint instead.XBoth \cite{dragan17robot} and \cite{knepper17implicit} are probabilistic in nature, hereYwe consider a worst-case model that is arguably more suitable for privacy considerations:KWe ask what an observer can plausibly infer via the history of its receivedBobservations.  In doing so, we are influenced by the philosophy ofMLaValle~\cite{lavalle2006planning}, following his use of the term informationJstate (I-state) to refer to a representation of information derived from adhistory of observations.\gobble{\footnotemark}  Finally, since parts of our stipulations may requirePconcealing information, we point out there is also recent work in deception (seeA\cite{masters17deceptive,dragan15deception}) and also obfuscation\cite{Wu2016Obfuscator}.K%TODO: Include citations to work we have discovered subsequently: Erdmann's-%Topological Privacy.  erdmann2017topologicalJ% - Anagha Kulkarni, Matthew Klenk, Shantanu Rane, Hamed Soroush, ResourceO% Bounded Secure Goal Obfuscation, AAAI Fall Symposium on Integrating Planning,>% Diagnosis, and Causal Reasoning, 2018.  kulkarni2018resourceM% - Anagha Kulkarni, Siddharth Srivastava and Subbarao Kambhampati, A UnifiedK% Framework for Planning in Adversarial and Cooperative Environments, ICAPS% workshop PlanRob, 2018.%kulkarni2018unifiedP% - Sarah Keren, Avigdor Gal, Erez Karpas, Privacy Preserving Plans in Partially&% Observable Environments, IJCAI, 2016%keren2016privacyO%\footnotetext{In his invited talk at WAFR'18, LaValle pointed out that his useL%of the term was based on earlier use by O. Morgenstern and J. von Neumann.}1\section{The model: worlds, robots and observers}\label{sec:preliminary}FFigure~\ref{fig:modeloverview} illustrates the three-way relationshipsJunderlying the setting we examine.  Most fundamentally, a robot executes aO\emph{plan} to achieve some goal in the \emph{world}, and the coupling of theseNtwo elements generates a stream of observations and actions. Both the plan andPthe action--observation stream are disclosed, though potentially only partially,Pto a third party, we term the \emph{observer}. The observer uses the stream, itsOknowledge of the plan, and also other known structure to infer properties aboutNthe interaction. Additionally, a stipulation is provided specifying particularOproperties that can be learned by the observer.  We formalize these elements inKterms of p-graphs and label maps (see~\cite{saberifar18pgraph}).%pgraphwafr\setlength{\textfloatsep}{10pt}\begin{SCfigure}[60]
\centering.\includegraphics[scale=0.7]{figure/model2.pdf}Q\caption{An overview of the setting: the robot is modeled abstractly as realizingQa plan to achieve some goal in the world and a third party observes, modeled as a=filter.  All three, the world, plan, and filter have concreterepresentations as p-graphs.%\label{fig:modeloverview}}\end{SCfigure}\vspace*{-20pt}1\subsection{P-graph and its interaction language}^We will start with the definition of p-graphs~\cite{saberifar18pgraph} and related properties:\begin{definition}[p-graph]NA \emph{p-graph} is an edge-labelled directed bipartite graph with $G=(V_y\cupV_u, Y, U, V_0)$, where \begin{tightenumerate}\item the finite vertex set 8$V(G)\defeq V_y\cup V_u$, whose elements are also called2\emph{states}, comprises two disjoint subsets: theM\emph{observation vertices} $V_y$ and the \emph{action vertices} $V_u$,% withG\item each edge $e$ originating at an observation vertex bears a set ofLobservations ${Y(e) \subseteq Y}$, containing \emph{observation labels}, andNleads to an action vertex, \item each edge $e$ originating at an action vertexMbears a set of actions ${U(e) \subseteq U}$, containing \emph{action labels},'and leads to an observation vertex, andN\item a non-empty set of states $V_0$ are designated as \emph{initial states},Ewhich may be either exclusively action states ($V_0\subseteq V_u$) or4exclusively observation states ($V_0\subseteq V_y$).\end{tightenumerate}\end{definition}NAn \emph{event} is an action or an observation. Respectively, they make up theHsets $U$ and $Y$, which are called the p-graph's \emph{action space} andX\emph{observation space}. We will also\gobble{ have occasion to} write $Y(G)$ and $U(G)$Jfor the observation space and action space of $G$. Though that is a slightKabuse of notation, the initial states will be written $V_0(G)$, similarly. NIntuitively, a p-graph abstractly represents a (potentially non-deterministic)Btransition system where transitions are either of type `action' orJ`observation' and these two alternate. The following definitions make this
idea precise.%\begin{definition}[event])%An event is an action or an observation.%\end{definition}%\vspace*{-3pt}"\begin{definition}[transitions to]JFor a given p-graph $G$ and two states $v,w\in V(G)$, a sequence of eventsS$\ell_1, \dots, \ell_k$ \emph{transitions in $G$ from $v$ to $w$} if there exists aMsequence of states $v_1, \dots, v_{k+1}$, such that $v_1=v$, $v_{k+1}=w$, andNfor each $i=1, \dots, k$, there exists an edge $v_i\xrightarrow{E_i}{v_{i+1}}$Ifor which $\ell_i\in E_i$, and $E_i$ is a subset of $Y(G)$ if $v_i$ is in2$V_y$, or a subset of $U(G)$ if $v_i$ is in $V_u$.\end{definition}PConcisely, we let the predicate $\trto{G}{v}{s}{w}$ hold if there is some way ofPtracing $s$ on $G$ from $v$ to $w$, i.e., it is \True iff $v$ transitions to $w$Gunder execution $s$.  Note, when $G$ has non-deterministic transitions,J$v$ may transition to multiple vertices under the same execution.  We only require that $w$ be one of them.%\vspace*{-3pt}7\begin{definition}[executions and interaction language]6%An execution on a p-graph $G$ is a sequence of eventsO%$s=\ell_0\ell_1\dots\ell_k$, if there exists some $v\in V_0(G)$ and some $w\inA%V(G)$ for which $\ell_0\dots\ell_k$ transitions from $v$ to $w$.KAn \emph{execution} on a p-graph $G$ is a finite sequence of events $s$, if>there exists some $v\in V_0(G)$ and some $w\in V(G)$ for whichD$\trto{G}{v}{s}{w}$.  The set of all executions on $G$ is called theM\emph{interaction language} (or, briefly, just \emph{language}) of $G$ and iswritten $\Language{G}$.\end{definition}OGiven any edge $e$, if $U(e)=L_e$ or $Y(e)=L_e$, we speak of $e$ \emph{bearing}the set $L_e$. *%\begin{definition}[interaction language] P%The set of all executions on $G$ is called the \emph{interaction language} (or,E%briefly, just \emph{language}) of $G$ and is written $\Language{G}$.%\label{def:lang}%\end{definition}%#\begin{definition}[joint-execution]TA \emph{joint-execution} on two p-graphs $G_1$ and $G_2$ is a sequence of events $s$>that is an execution of both $G_1$ and $G_2$, written as $s\inA\Language{G_1}\cap \Language{G_2}$. The p-graph producing all theNjoint-executions of $G_1$ and $G_2$ is their tensor product graph with initialEstates $V_0(G_1) \times V_0(G_2)$, which we denote $G_1\pgprod G_1$. \end{definition}OA vertex from $G_1\pgprod G_2$ is as a pair $(v_1, v_2)$, where $v_1\in V(G_1)$Oand $v_2\in V(G_2)$. Next, the relationship between the executions and verticesis established.%\begin{definition}%[reaching]PThe set of vertices reached by execution $s$ in $G$, denoted $\reachedv{G}{s}$, Hare the vertices to which the execution $s\in \Language{G}$ transitions,Jstarting at an initial state. Symbolically, $\reachedv{G}{s} \defeq\{v \inLV(G)\;|\;\exists v_0 \in V_0(G), \trto{G}{v_0}{s}{v}\}.$ Further, the set of3executions reaching vertex $v$ in $G$ is written asG$\reachings{G}{v}\defeq\{s\in \Language{G}\,|\,v\in \reachedv{G}{s}\}$.\end{definition}w%% Note to Yulin: The paper_closest_to_conference.tex contains a longer description there, that jason cut with \gobble.�%%                Please look at the current version, and that version, and see if the longer version is better (assuming we have the space)PThe naming here serving to remind that $\mathcal{V}$ describes sets of vertices,5$\mathcal{S}$ describes sets of strings/executions.  HThe collection of sets $\{\reachings{G}{v_0}, \reachings{G}{v_1}, \dots,F\reachings{G}{v_i}\dots\}$ can be used to form an equivalence relationK$\veq{}G{}$ over executions, under which $\veq{s_1}{G}{s_2}$ if and only ifW\mbox{$\reachedv{G}{s_1}=\reachedv{G}{s_2}$}.  This equivalence relation partitions theIexecutions in $\Language{G}$ into a set of non-empty equivalence classes:M$\Language{G}/{\veq{}{G}{}}=\{[r_0]_G, [r_1]_G, [r_2]_G, \dots\}$, where eachRequivalence class is $[r_i]_G=\{s\in \Language{G}\,|\,\veq{r_i}{G}{s}\}$ and $r_i$Lis a representative execution in $[r_i]_{G}$.  The intuition is that any twoMexecutions that transition to identical sets of vertices are, in an importantsense, indistinguishable.1We shall consider systems where the vertices of a?p-graph constitute the state that is stored, acted upon, and/or?represented---they are, thus, akin to a `sufficient statistic'.%$\begin{definition}[state-determined]DA p-graph $G$ is in a \emph{state-determined} presentation, or is in>\emph{state-determined} form, if $\,\forall s\in \Language{G},|\reachedv{G}{s}|=1$.  \end{definition}LThe procedure to expand any p-graph $G$ into a state-determined presentationC$\sde{G}$ can be found in Algorithm 2 of~\cite{saberifar18pgraph}. C%The key is to make sure that there is only a single starting stateN%and that the labels on different outgoing edges of the same vertex have empty%intersections. <The language of p-graphs is not affected by state-determined3expansion, i.e., $\Language{G}=\Language{\sde{G}}$.M%Since the equivalence class contains the executions that reach exactly (i.e.J%reach and reach only) $\mathcal{V}^{G}_{r_i}$, we have $[r_i]=(\cap_{v\in?%\mathcal{V}^{G}_{r_i}} \mathcal{S}^{G}_{v})\setminus\cup_{v\in<%(V(G)\setminus\mathcal{V}^{G}_{r_i})} \mathcal{S}^{G}_{v}$.MNext, one may start with vertices and ask about the executions reaching thoseMvertices. (Later, this will be part of how an observer makes inferences aboutthe world.)L%TODO: The sentence be: It is easy to show that $[s]$ is the intersection ofP%those sets $\reachings{G}{v}$ containing $s$.  Yulin: We both changed it to fix.%it. Your idea is in the comment above, but itR%doesn't really connect with what was there. I've put my proposal back. Is there aX% place where we need to talk about [s]? Otherwise, I'm not sure we need what you wrote.F% I see you point. I think the current sentence is the right answer. IO% previously thought you want to talk about the executions. For [s], I think it% is clear enough below.\begin{definition}\label{vertextoequiv}FGiven any set of vertices $B\subseteq V(G)$ in p-graph $G$, the set of@executions that reach exactly (i.e. reach and reach only) $B$ isP$\exactreachings{G}{B}\defeq(\cap_{v\in B} \reachings{G}{v})\setminus \cup_{v\in%(V(G)\setminus B)} \reachings{G}{v}$.�%If $\mathbb{S}^{G}_{B}\neq \emptyset$, then $\mathbb{S}^{G}_{B}$ is an equivalence class. If $\mathbb{S}^{G}_{B}=\emptyset$, then there is no execution $s\in \Language{G}$, such that $\mathcal{V}^{G}_s=B$.\end{definition}LAbove, the $\cap_{v\in B} \reachings{G}{v}$ represents the set of executionsKthat reach every vertex in $B$. By subtracting the ones that also reach theMvertices outside $B$, $\exactreachings{G}{B}$ describes the set of executionsPthat reach exactly $B$. In Figure~\ref{fig:stringbehindvertices}, the executionsNreaching $w_3$ are represented as $\reachings{G}{w_3}=\{a_1o_1, a_2o_1\}$. But7the executions reaching and reaching only $\{w_3\}$ areK$\exactreachings{G}{\{w_3\}}=\{a_1o_1\}$ since $a_2o_1$ also reaches $w_4$.PSpecifically, the equivalence class $[r_i]_G$ contains the executions that reach'exactly $\reachedv{G}{r_i}$, so we have0$[r_i]_G=\exactreachings{G}{\reachedv{G}{r_i}}$.\vspace*{-15pt}\begin{SCfigure}[40]L{\includegraphics[scale=0.85]{figure/stringbehindvertices.pdf}}\hspace{20pt}I\caption{An example showing the difference between `reaches' and `reaches?exactly' as distinguished in notation as $\reachings{G}{w}$ and=$\exactreachings{G}{\{w\}}$.\label{fig:stringbehindvertices}}
\vspace*{2pt}\end{SCfigure}\vspace*{-20pt}%\vspace*{-6pt}(\subsection{Planning problems and plans}%\vspace*{-3pt}DIn the p-graph formalism, planning problems and plans are defined as!follows~\cite{saberifar18pgraph}./\begin{definition}[planning problems and plans]PA \emph{planning problem} is a p-graph $W$ along with a {goal region} $V_{\goal}@\subseteq V(W)$; a \emph{plan} is a p-graph $P$ equipped with a 0{termination region} $V_{\term} \subseteq V(P)$.\end{definition}PPlanning problem $(W, V_{\goal})$ is solved by some plan $(P, V_{\term})$ if theKplan always terminates (i.e., reaches $V_{\term}$) and only terminates at agoal. Said with more precision:\begin{definition}[solves]\label{def:solves}PA plan $(P,V_{\term})$ \emph{solves} a planning problem $(W,V_{\goal})$ if thereIis some integer which bounds length of all joint-executions, and for eachOjoint-execution and any pair of nodes $(v \in V(P),w \in V(W))$ reached by that8execution simultaneously, the following conditions hold:\begin{tightenumerate}M\item if $v$ and $w$ are both action nodes and, for every label borne by eachNedge originating at $v$, there exist edges originating at $w$ bearing the sameLaction label; \item if $v$ and $w$ are both observation nodes and, for everyMlabel borne by each edge originating at $w$, there exist edges originating at($v$ bearing the same observation~label; -% had to be sneaky to stop it wrapping a line6\item if $v \in V_{\term}$ and then $w \in V_{\goal}$;H\item if $v \notin V_{\term}$ then some extended joint-execution exists,Dcontinuing from $v$ and $w$, that does reach the termination region.\end{tightenumerate}\end{definition}NIn the above, properties 1) and 2) describe a notion of safety; property 3) ofOcorrectness; and 4) of liveness.  In the previous definition, there is an upperEbound on joint-execution length.  We say that plan $(P,V_{\term})$ isA\emph{$c$-bounded} if, $\forall s\in \Language{P}$, $|s|\leq c$. %\vspace*{-6pt}G\subsection{Information disclosure policy, divulged plan, and observer}%\vspace*{-4pt}]The agent who is the observer sees a stream of the robot's actions and observations, and usesIthem to build estimates (or to compute general properties) of the robot'sNinteraction with the world. But the observer's access to this information willGusually be imperfect---either by design, as a consequence of real-worldLimperfections, or some combination of both.  Conceptually, this is a form ofKpartial observability in which the stream of symbols emitted as part of theLrobot's execution is distorted into to the symbols seen by the observer (seeFFigure~\ref{fig:dataflow}).  For example, if some pairs of actions areMindistinguishable from the perspective of the observer, this may be expressedLwith a function that maps those pairs of actions to the same value.  In thisLpaper, this barrier is what we have been referring to (informally, thus far)Kwith the phrase \emph{information disclosure policy}. It is formalized as aMmapping from the events in the robot's true execution in the world p-graph to&the events received by the observer.  \vspace*{-6pt}\begin{SCfigure} 	\centering5	\includegraphics[scale=0.73]{figure/informationflow}I\caption{The information disclosure policy, divulged plan and informationJstipulation.  Even when the observer is a strong adversary, the disclosureLpolicy and divulged plan can limit the observer's capabilities effectively.}\label{fig:dataflow}
\vspace*{7pt}\end{SCfigure}\vspace*{-18pt}\vspace*{-2pt}1\begin{definition}[Information disclosure policy]EAn \emph{information disclosure policy} is a label map $h$ on p-graph%$G=(V_u\cup Y_u, Y, U, V_0)$G$G$, mapping from elements in the combined observation and action space*$Y(G)\cup U(G)$ to some set of events $X$.\end{definition}\vspace*{-2pt}LThe word `policy' hints at two interpretations: first, as something given asNa predetermined arrangement (that is, as a rule); secondly, as something to beFsought (together with a plan).  Both senses apply in the present work;Nthe exact transformation describing the disclosure of information will be usedPfirst (in Section~\ref{section:seekplan}) as a specification and then, later (inNSection~\ref{section:seekplanlabelmap}) as something which planning algorithmsNcan produce. How the information disclosure policy is realized in some settingLdepends on which sense is apt: it can be interpreted as describing observersF(showing that for those observers unable to tell $y_i$ from $y_j$, theLstipulations can be met), or it can inform robot operation (the stipulationsNrequire that the robot obfuscate $u_\ell$ and $u_m$ via means such as explicit1concealment, sleight-of-hand, misdirection, etc.)MThe observer, in addition, may also have imperfect knowledge of robot's plan,Pwhich is leaked or communicated from the side-channel. The \emph{disclosed plan}Ois also modeled as a p-graph, which may be weaker than knowing the actual plan.FA variety of different types of divulged plan are introduced later (inPSection~\ref{section:divulgedplan}) to model different prior knowledge availableOto an observer; as we will show, despite their differences, they can be treatedin a single unified way.PThe next step is to provide formal definitions for the ideas just described.  InLthe following, we refer to $h$ as the map from the set $Y\cup U$ to some setJ$X$, and refer to its preimage $\inv{h}$ as the map from $X$ to subsets ofI$Y\cup U$. The notation for a label map $h$ and its preimage $\inv{h}$ isLextended in the usual way to sequences and sets: we consider sets of events,Oexecutions (being sequences), and sets of executions. They are also extended toCp-graphs in the obvious way, by applying the function to all edges.!% <zyl> due to space limit </zyl>%\begin{description}I%\item[Events] Given any set of events $L\subseteq Y\cup U$, its image isO%$h[L]=\{h(\ell)\;|\;\ell\in L\}$. And conversely, for set $L'\subseteq X$, itsB%preimage is $\inv{h}[L']=\{\ell \in Y\cup U\;|\;h(\ell)\in L'\}$.%s%\item[Executions] Given any execution $s=\ell_0\ell_1\dots\ell_k$, where $\ell_i\in Y\cup U$, %$(0\leq i\leq k)$, 7%its image is $h(s)=h(\ell_0)h(\ell_1)\dots h(\ell_k)$,c%and for any execution $s'=\ell'_0\ell'_1\dots\ell'_k$, where $\ell'_i\in X$, % $(0\leq i\leq k)$, 1%its preimage is $\inv{h}(s')=\{s\;|\;h(s)=s'\}$.%O%\item[Sets of executions] Given any set of executions $A$, where $\forall s\inQ%A, s\in \cl{(Y\cup U)}$, its image is $h[A]=\{h(s)\;|\;s\in A\}$. Conversely forQ%any set of executions $A'$, where $\forall s'\in A', s'\in \cl{X}$, its preimage(%is $\inv{h}[A']=\{s\;|\;h(s)\in A' \}$.%J%\item[P-graphs] Given any p-graph $G=(V_u\cup Y_u, Y, U, V_0)$, its imageK%p-graph $h\langle G\rangle=(V_u\cup Y_u,  h[Y], h[U], V_0)$ is produced byL%replacing the set of events $L$ on each edge $e$ with $h[L]$.  Analogously,E%given p-graph $G=(V_u\cup Y_u, X_y, X_u, V_0)$, its preimage p-graphM%$\inv{h}\langle G\rangle=(V_u\cup Y_u,  \inv{h}[X_y], \inv{h}[X_u], V_0)$ isF%constructed by replacing the set of events $L'$ on each edge $e$ with%$\inv{h}[L']$.%\end{description}L%<jmo>The name "I-state" is not really explained anywhere in this paper, andO%I'm not certain that explaining it here will be helpful.  Perhaps these should?%just be called "observer graphs" or "observer p-graphs".</jmo>R%This is a pretty substantial change, which I'd want to change back once we'd get D% more space. Instead, I wove I-state into the related work section.IFor brevity's sake, the outputs of~$h$ will be referred to simply as `theFimage space.' The function $h$ may either preserve information (when aCbijection) or lose information (with multiple inputs mapped to one 	output). SThe loss of information is felt in $Y\cup U$ by the extent to which some element ofG$Y \cup U$ grows under $\inv{h} \of h$, and for all  $\ell\in Y\cup U$,N$\inv{h}\of h(\ell)\supseteq \{\ell\}$.  In contrast, starting from $x \in X$,Lthe uncertainty, apparent via set cardinality under $\inv{h}$, is washed outJagain when pushed forward to the image space $X$ via $h\of\inv{h}$,  i.e.,)$\forall x\in X$, $h\of\inv{h}(x)=\{x\}$.Q%The outputs of~$h$, which,  for sake of brevity, we will refer to simply as `theP%image space'.  Given any event generated from the world, the observer may inferK%the potential events that could happen from the image of the actual event.O%Similarly, we are also interested in the inference starting from the events inI%the image space. The results are formalized with the following lemmas:} %\begin{lemma}%\label{lemma:hinvh}K%For any event $\ell \in Y\cup U$, $\inv{h} \of h(\ell)\supseteq \{\ell\}$.Q%Similarly, we have $\forall L\subseteq Y\cup U, \inv{h} \of h[L]\supseteq L$ andO%$\forall s=\ell_0\ell_1\dots\ell_n\in \cl{(Y\cup U)}$, $s\in \inv{h}\of h(s)$.%\end{lemma}%\begin{proof}F%First, we are going to prove $\forall \ell \in Y\cup U$, $\inv{h} \ofQ%h(\ell)\supseteq \{\ell\}$. Since $h$ is a function, there are two cases for theI%images of the events in $G$: First, $\forall \ell_1, \ell_2\in U\cup Y$,N%$h(\ell_1)\neq h(\ell_2)$. In this case, no two events are mapped to the sameB%output. In other words, each image element has a unique preimage,M%$\{\ell\}=\inv{h}(h(\ell))$.  Secondly, $\exists \ell_1, \ell_2\in {U(G)\cup,%Y(G)}$, $h(\ell_1)=h(\ell_2)$. Then we haveN%$\inv{h}(h(\ell_1))=\inv{h}(h(\ell_2))=\ell_1\cup \ell_2$, $\{\ell_1\}\subsetH%\inv{h}(h(\ell_1))$, and $\{\ell_2\}\subset \inv{h}(h(\ell_2))$. Hence,(%$\inv{h}\of h(\ell)\supseteq \{\ell\}$.%O%Next, following the result of $\inv{h} \of h(\ell)\supseteq \{\ell\}$, we haveP%that $\inv{h} \of h[L]=\cup_{\ell_i\in L}{\inv{h}\of h[\ell_i]\supseteq L}$ for%any $L\subseteq {Y\cup U}$.%C%Finally, we will prove $s\in \inv{h}\of h(s)$ by induction for all3%$s=\ell_0\ell_1\dots\ell_n\in \cl{(Y\cup U)}$. LetP%$s^{k}=\ell_0\ell_{1}\dots\ell_k$ be the prefix of $s$ with length $k+1$, whereQ%$0\leq k< n$. When $k=0$, $s^{0}$ only contains an action or observation and, we0%have ${s^{0}\in \inv{h}\of h(s^{0})}$.  SupposeM%$s^{k}=\ell_0\ell_{1}\dots\ell_k\in \inv{h}\of h(s^{k})$ holds for $k$.  TheQ%inductive step: ${\inv{h}\of h(s^{k+1})}= {\cup_{\ell'_0\ell'_{1}\dots\ell'_k\inG%\inv{h}\of h(s^{k})}} {\cup_{\ell'_{k+1}\in \inv{h}\of h(\ell_{k+1})}}Q%\ell'_0\ell'_{1}\dots\ell'_k\ell'_{k+1}\ni \ell_0\ell_{1}\dots\ell_{k+1}$, whichM%is since $\ell_0\ell_1\dots\ell_k\in \inv{h}\of h(s^{k})$ and $\ell_{k+1}\inN%\inv{h}\of h(s^{k+1})$. Hence, $s_{k+1}\in \inv{h}\of h(s^{k+1})$. Therefore,E%$s\in \inv{h}\of h(s), \forall s\in \cl{(Y\cup U)}$.\qed \end{proof}%%\begin{lemma}%\label{lemma:hhinv}J%For any $\ell'\in X$, $h\of \inv{h}(\ell')=\{\ell'\}$. Similarly, we have<%$\forall L'\subseteq X$, $h\of \inv{h}[L']=L'$ and $\forallO%s=\ell'_0\ell'_1\dots\ell'_k\in \cl{X}$, $h\of \inv{h}(s)=\{s\}$.  \end{lemma}%\begin{proof}O%Firstly, we will prove $h\of \inv{h}(\ell')=\{\ell'\}$ holds for any $\ell'\inN%X$. Let $\inv{h}(\ell')=\{l\in Y\cup U | h(l)=\ell'\}$. Then $\forall \ell\in;%\inv{h}(\ell')$, we have $h(\ell)=\ell'$. Therefore, $h\of%\inv{h}(\ell')=\{\ell'\}$.%O%Following from $h\of \inv{h}(\ell')=\{l'\}$, we have $h\of \inv{h}[L']=L'$ for%any $L'\subseteq X$.%D%Thirdly, we will prove $h\of \inv{h}(s)=\{s\}$ by induction for anyP%$s=\ell'_0\ell'_1\dots\ell'_n\in \cl{X}$. Let $s^{k}$ be the prefix of $s$ withP%length $k+1$, where $0\leq k<n$. When $k=0$, $s^{0}$ only contains an action orB%observation and, we have $\{s^{0}\}=h\of \inv{h}(s^{0})$. SupposeK%$\{s^{k}\}=h\of \inv{h}(s^{k})$ holds for $k$. Then $h\of \inv{h}(s^{k+1})?%=\cup_{\ell''_0\ell''_{1}\dots\ell''_k\in h\of \inv{h}(s^{k})}2%\cup_{\ell''_{k+1}\in h\of \inv{h}(\ell''_{k+1})}P%\ell''_0\ell''_{1}\dots\ell''_{k+1}=\{\ell'_0\ell'_1\dots\ell'_{k+1}\}$. Hence,K%$h\of \inv{h}(s^{k+1})=\{s^{k+1}\}$. Therefore, $\forall s\in \cl{X}, h\of%\inv{h}(s)=\{s\}$.\qed %\end{proof}\vspace*{-2pt}!\begin{definition}[I-state graph]JFor planning problem $(W,V_{\goal})$, plan $(P,V_{\term})$ and informationNdisclosure policy $h: Y(W)\cup U(W) \to X$, an observer's \emph{I-state graph}P$I$ is a p-graph, whose inputs are from the image space of  $h$ (i.e., $Y(I)\cup:U(I)=X$), with $\Language{I} \supseteq h[\Language{W}]$.  e%that is safe on the joint-execution of the world and plan p-graphs $\Language{W} \cap \Language{P}$.PThe action space and observation space of $I$ are also written as $X_u=U(I)$ and$X_y=Y(I)$. \end{definition}PInherited from the property of $h\of\inv{h}$, for any I-state graph $I$, we have>$I=h\of\inv{h}\langle I\rangle$, and $\forall B\subseteq V(I),L\inv{h}[\exactreachings{I}{B}]=\exactreachings{\inv{h}\langle I\rangle}{B}$.O% <jmo>This bit seems suspect: "is safe on the joint-execution of the world andP% plan p-graphs".</jmo> <das>yes, it was corrected to "joint-execution" then the%% detail in the sentence below added.PThe observer's I-state graph is a p-graph with events in the image space $X$. ByNhaving $\Language{I} \supseteq h[\Language{W}]$, we are requiring that strings3generated in the world can be safely traced on $I$.G%By requiring safety, we want $\Language{I} \supseteq h[\Language{W}]$.p%Whether $X_u$ and $X_y$ are disjoint or not will depends on $U$ and $Y$ being disjointed and also $h$'s role.  M% This doesn't need a new definition as it is using the notation given above:%\begin{definition}%\label{def:invpgraph}O%Given an I-state graph $I=(V_y\cup V_u, X_y, X_u, V_0)$, its counterpart afterP%replacing every set of events in the image space $L'\subseteq X_u\cup X_y$ with@%$\inv{h}[L']$ is written $\inv{h}\langle I\rangle=(V_y\cup V_u,%%\inv{h}[X_u], \inv{h}[X_y], V_0)$.  %\end{definition}Q%Using the notation above, we will frequently speak of $\inv{h}\langle I\rangle$.G%Next, are some basic properties of the vertices and executions of $I$.%\vspace*{-2pt}:%\begin{restatable}[Properties of $I$]{lemma}{istateprops}%\label{lem:istateprops}'%Given $I$ and $h$, the following hold:K%%Then vertices and executions in $I$, $\inv{h}\langle I\rangle$, and $h\of8%%\inv{h}\langle I\rangle$ has the following properties:%\vspace*{-2pt}%\begin{enumerate})%\renewcommand{\theenumi}{\roman{enumi}}%(%\item $I=h\of \inv{h}\langle I\rangle$. %\label{item:handpreh}\\[-12pt] %>%\item $\forall s'\in \Language{I}, \forall s\in \inv{h}(s')$,-%$\reachedv{I}{s'}=\reachedv{\inv{h}\langle I%\rangle}{s}$.%\label{item:iso}\\[-12pt]%A%\item $\Language{h^{-1}\langle I\rangle}=\inv{h}[\Language{I}]$.%\label{item:moveh}\\[-12pt] % %\item $\forall B\subseteq V(I),>%\inv{h}[\exactreachings{I}{B}]=\exactreachings{\inv{h}\langle%I\rangle}{B}$.%\label{item:reachings}%\end{enumerate}%\end{restatable}%\vspace*{-4pt}-%\begin{proof}[Property~\ref{item:handpreh}.]%\label{suppl:istatepropproofs}Q%According to Lemma \ref{lemma:hhinv}, each event set $L$ in $I$ will  not changeN%when we apply operation $h\of \inv{h}$ on $I$. Therefore, we have $I=h\langleQ%\inv{h}\langle I\rangle\rangle$ by replacing every set of events $L$ in $I$ with#%$h\of \inv{h}[L]$.\qed \end{proof}%(%\begin{proof}[Property~\ref{item:iso}.]:%We need to prove that $s'$ and its preimage $s$ reach theN%same\footnotemark\xspace set of vertices in $I$ and $\inv{h}\langle I\rangle$M%respectively. According to the construction of $\inv{h}\langle I\rangle$, we<%have $\forall s'\in \Language{I}, \forall s\in \inv{h}(s'),M%\reachedv{I}{s'}\subseteq \reachedv{\inv{h}\langle I \rangle}{s}$.  Next, weC%will prove $\forall s'\in \Language{I}, \forall s\in \inv{h}(s')$,F%$\reachedv{I}{s'}\supseteq \reachedv{\inv{h}\langle I \rangle}{s}$ byN%contradiction. Suppose $\exists s'\in \Language{I}, \exists s\in \inv{h}(s'),O%\reachedv{I}{s'}\not\supseteq \reachedv{\inv{h}\langle I \rangle}{s}$. Then weP%have $\reachedv{I}{s'}\subset \reachedv{\inv{h}\langle I\rangle}{s}$. If $s$ isQ%the preimage of only $s'$, then we should have, according to the construction ofF%$\inv{h}\langle I\rangle$, $\reachedv{I}{s'}=\reachedv{\inv{h}\langleL%I\rangle}{s}$ instead. Hence, $s$ is the preimage of at least two differentI%executions $s'$ and $s''$, which contradicts with the fact that $h$ is aL%function. Therefore, $\forall s'\in \Language{I}, \forall s\in \inv{h}(s'),:%\reachedv{I}{s'}=\reachedv{\inv{h}\langle I \rangle}{s}$.%Q%\footnotetext{Since, by $\inv{h}\langle I\rangle$ we refer to the graph $I$ withO%each of the edge labels replaced by preimages under $h$, there is a one-to-oneQ%correspondence between the two graphs via a natural isomorphism. For convenienceN%we speak of the `same' vertex rather than being explicit about the associatedC%bijection and, further, we have used `$=$' rather than `$\cong$'.}%%P%$\forall s'\in \Language{I}, \forall s\in \inv{h}(s')$, $s'$ reach the same setN%of vertices as those reached by $s$ in $\inv{h}\langle I \rangle$. Since each>%vertex in $\reachedv{I}{s'}$ is isomorphic to the same one inH%$\reachedv{\inv{h}\langle I \rangle}{s}$, we have $\reachedv{I}{s'}$ is;%identical to $\reachedv{\inv{h}\langle I \rangle}{s}$.\qed%\end{proof}*%\begin{proof}[Property~\ref{item:moveh}.]O%$\implies$: Given any execution $s$ from p-graph $\inv{h}\langle I\rangle$, weB%will prove $h(s)$ is an execution from p-graph $I$.  According toF%Lemma~\ref{lem:istateprops}.\textit{\ref{item:handpreh}}, $I=h\langleN%\inv{h}\langle I\rangle \rangle$. Thus, $h(s)$ is an execution on $I=h\langleH%\inv{h}\langle I\rangle \rangle$. And we have $\Language{\inv{h}\langle+%I\rangle}\subseteq \inv{h}[\Language{I}]$.%N%$\impliedby$: Given any execution $s\in \inv{h}[\Language{I}]$, we will proveQ%$s\in \Language{\inv{h}\langle I\rangle}$. For any $s\in \inv{h}[\Language{I}]$,>%we have $h(s)$ is an execution from p-graph $I$. According toQ%Lemma~\ref{lem:istateprops}.\textit{\ref{item:iso}}. the set of vertices reachedP%by $h(s)$ in p-graph $I$ is isomorphic to the set of vertices reached by $s'\inK%\inv{h}(h(s))$ in $\inv{h}\langle I\rangle$. Hence, $s$ is an execution in@%$\inv{h}\langle I\rangle$. Therefore, $\Language{\inv{h}\langle0%I\rangle}\supseteq \inv{h}[\Language{I}]$.\qed %\end{proof}%.%\begin{proof}[Property~\ref{item:reachings}.]O%$\implies$: Given any execution $s\in \inv{h}[\exactreachings{I}{B}]$, then weN%have $h(s)\in \exactreachings{I}{B}$ and $\reachedv{I}{h(s)}=B$. According to=%Lemma~\ref{lem:istateprops}.\textit{\ref{item:iso}}, we haveL%$\reachedv{\inv{h}\langle I\rangle}{s}=\reachedv{I}{h(s)}=B$.  Hence, $s\in.%\exactreachings{\inv{h}\langle I\rangle}{B}$.%G%$\impliedby$: Given any execution $s\in \exactreachings{\inv{h}\langleQ%I\rangle}{B}$, then we have $\reachedv{\inv{h}\langle I\rangle}{s}=B$. According8%to Lemma~\ref{lem:istateprops}.\textit{\ref{item:iso}},N%$\reachedv{I}{h(s)}=\reachedv{\inv{h}\langle I\rangle}{s}=B$. Hence, $h(s)\in8%\exactreachings{I}{B}$ and therefore, we now have $s\in&%\inv{h}[\exactreachings{I}{B}]$.\qed %\end{proof}?%\noindent \textit{Proofs appear in the supplementary material,=%Section~\ref{suppl:istatepropproofs}.}\medskip \begin{proof}-%Proofs appear in the supplementary material,8%Section~\ref{suppl:istatepropproofs}.  \qed \end{proof}N%Using the notation we have already given above, we will write $\inv{h}\langleN%I\rangle$ for $(V_y\cup V_u, \inv{h}[X_u], \inv{h}[X_y], V_0)$, which is justH%the counterpart of I-state graph $I=(V_y\cup V_u, X_y, X_u, V_0)$ afterP%replacement of every set of events in the image space $L'\subseteq X_u\cup X_y$%by $\inv{h}[L']$.2Next, we formalize the crucial connection from theOinteraction of the robot and world, via the stream of symbols generated, to theWstate tracked by the observer.  Inference proceeds from the observer back to the world,1though causality runs the other way (glance againKat~Figure~\ref{fig:modeloverview}). We begin, accordingly, with that latter
direction.+\begin{definition}[compatible world states]!\label{def:compatibleworldstates}LGiven observer I-state graph $I$, robot's plan $(P, V_{\term})$, world graphM$(W, V_{\goal})$, and label map $h$, the world state $w$ is \emph{compatible}Nwith the set of I-states $B\subseteq V(I)$ if $\exists s\in \Language{W}$ suchLthat $s\in \underset{\strut(1)}{\underbrace{\inv{h}[\exactreachings{I}{B}]}}9\cap \underset{\strut(2)}{\underbrace{\Language{P}}} \cap5\underset{\strut(3)}{\underbrace{\reachings{W}{w}}}$.\end{definition}\vspace*{-16pt}HInformally, each of the three terms can be interpreted as:\vspace*{-3pt}%\begin{enumerate}[(1)]P\item An observer with I-state graph $I$ may ask which sequences are responsible8for having arrived at states $B$.  The answer is the setN$\exactreachings{I}{B}$, being the executions contained in equivalence classesOthat are indistinguishable up to states in~$I$.  Those strings are in the imageJspace $X$, so, to obtain an answer in $Y\cup U$, we take their preimages. =Every execution in $\inv{h}[\exactreachings{I}{B}]$ leads theobserver to $B$. Note that9information may be degraded by either $h$, $I$, or both. \gobble{#Figure~\ref{suppl:fig:infocollapse}Oprovides a visual example that shows how information can be degraded by a labelPmap $h$, an I-state graph $I$, and both together.  The first gives a scenario byHproviding a world p-graph $W$, a plan $P$, and divulged plan informationM$D$---all three are identical.  The second figure shows an I-state graph withNthe same structure as $W$ and an identity label map. Every I-state correspondsNto a single world state in this case. In the third figure, there is an I-stateHgraph with the same structure as $W$, thus clearly possessing sufficientNstructure to account for the world states. But here a label map conflates someOactions and some observations. A consequence is that the world states $w_1$ andP$w_2$ are indistinguishable given I-state $i_1$ and plan $P$. In the last figureKboth $h$ and $I$ degrade information and do so independently. In this case,M$w_3$ and $w_4$ are indistinguishable owing to the label map, $w_5$ and $w_6$?are indistinguishable owing to the collapsed structure in $I$.}N%(Note that information is degraded both by $h$ and~$I$, an example to clarifyI%this appears in Figure~\ref{suppl:fig:infocollapse} in the supplementary%materials.)\gobble{\begin{figure}[t!]
\centering5\includegraphics[scale=0.78]{figure/infocollapse.pdf}K\caption{Both the label map and the  I-state graph can degrade information.OLeftmost:~a scenario where the world p-graph $W$, a plan $P$, and divulged planNinformation $D$ are all identical.  Second from the left: an I-state graph $I$Nwith the same structure as $W$ and an identity label map $h$.  Second from theNright: an I-state graph with $I$ the same structure as $W$ and a label map $h$?which conflates some actions/observations.  Rightmost:~both $h$Jand $I$ degrade information independently.  \label{suppl:fig:infocollapse}}\vspace*{-10pt}\end{figure}}O\item The set of executions that may be executed by the robot is represented byP$\Language{P}$. If the observer knows that the robot's plan takes, say, the highMroad, this information allows the observer to remove executions involving therobot along the low road. F\item The set of executions reaching world state $w$ is represented byE$\reachings{W}{w}$.  Two world states $w, w'\in V(W)$ are essentiallyPindiscernible or indistinguishable if ${\reachings{W}{w}=\reachings{W}{w'}}$, as8the sets capture the intrinsic uncertainty of world $W$.\end{enumerate}AWhen an observer is in $B$, and $w$ is compatible with $B$, thereNexists some execution, a certificate, that the world could plausibly be in $w$Osubject to (1)~the current information summarized in $I$; (2)~the robot's plan;K(3)~the structure of the world. The set of \emph{all} world states that areOcompatible with $B$ is denoted $\compatablew{I, P}{B}$, which is the observer'sOestimate of the world states when known information about $W$, $P$ and $I$ haveall been incorporated.%\begin{figure}[h!]%\centering1%\includegraphics[scale=0.7]{figure/infocollapse}?%\caption{An example showing how information can be degraded byO%a label map $h$, an I-state graph $I$, and both together. The first (leftmost)L%figure gives a world p-graph $W$, a plan $P$, and divulged plan informationN%$D$.  The second figure shows an I-state graph with the same structure as $W$P%and an identity label map. Every I-state corresponds to a single world state inH%this case. In the third figure, there is an I-state graph with the sameN%structure as $W$, thus clearly possessing sufficient structure to account forO%the world states. But a label map has conflated some actions/observations and,L%consequently, $w_1$ and $w_2$ are indistinguishable given I-state $i_1$ andI%plan $P$. The last figure shows both $h$ and $I$ may degrade informationP%independently. In this case, $w_3$ and $w_4$ are indistinguishable owing to the%label map, $w_5$ and $w_6$ are;%indistinguishable owing to the collapsed structure in $I$.-%\label{suppl:fig:infocollapse}} \end{figure}KA typical observer may know less about the robot's future behavior than theGrobot's full plan. Weaker knowledge of how the robot will behave can beIexpressed in terms of some p-graph $D$, such that ${\Language{D}\supseteqP\Language{P}}$.  (Here the mnemonic is that it is the divulged information aboutIthe robot's plan, which one might imagine as leaked or communicated via aside-channel.):Notice that the information divulged to the observer about/the robot's execution is in the preimage space. The key reason for this modelingJdecision is that information may be lost under label map $h$; an observer Wgains the greatest information when the plan is disclosed in the preimage space and, aswe consider worst-case5conditions, we are interested in what the strongest  M(even adversarial) observers might infer. Thus, we study divulgence where the%observer obtains as much as possible.Q%\changed{Another decision is that the information divulged to the observer aboutQ%the robot's execution is in the preimage space.  This modeling decision may seemP%strange at first blush so we provide some explanation and justification for it.J%As the observer will only see things in the image space, it may seem thatM%granting access to information in the preimage or the image space would haveO%little difference. But, since inference occurs by pulling back observed eventsO%to preimage space and then taking an intersection, there can be an appreciableP%difference. As this paper is interested in a worst-case adversarial conditions,N%we are interested in what strong ne'er-do-well observers might infer and thusL%study the problem where the adversary gains the maximum possible. From thisK%setting the other variant can be easily posed as well (one simply needs toQ%consider $\inv{h}\of h\langle D\rangle$ to simulate the knowledge of image space%information).}
%\changed{K%To clarify the previous statements, we formalize the image space inference
%process:}%%\begin{definition}L%Given an I-state graph $I$, divulged information about the robot's behaviorP%$h\langle D\rangle$ in the image space, and label map $h$, the set of estimatedI%world states for I-states $B\subseteq V(I)$ is $\compatablew{I, h\langle>%D\rangle}{B}=\{w\in V(W)| \inv{h}[\exactreachings{I}{B}] \capP%\Language{\inv{h}\of h\langle D\rangle} \cap \reachings{W}{w}\neq \emptyset\}$.%\end{definition}%%\begin{lemma}%\label{lemma:hinvhpgraph}L%Given any p-graph $D$, $\Language{D}\subseteq \Language{\inv{h}\of h\langle%D\rangle}$.  %\end{lemma}%\begin{proof}P%According to Lemma \ref{lemma:hinvh}, for event $\ell\in \Language{D}$, we haveK%$\{\ell\}\subseteq \Language{\inv{h} \of h(\ell)}$. Then the set of eventsO%bearing in each edge of p-graph $\inv{h}\of \langle D\rangle$ is a superset ofI%the corresponding edge in p-graph $D$. Therefore, $\Language{D}\subseteq/%\Language{\inv{h}\of h\langle D\rangle}$.\qed %\end{proof}%%%\begin{figure}[t]%\centering<%\includegraphics[scale=0.7]{figure/disclosedpreimgplan.pdf}M%\caption{Find the estimated world states when given world graph $W$, I-stateJ%graph $I$, divulged graph $D$ or $h\langle D\rangle$, label map $h=\{a_1,@%a_2\mapsto a;  o_1,o_2\mapsto o\}$.\label{fig:relaxedbehavior}}%\vspace*{-5pt}
%\end{figure}%%\begin{theorem} %\label{theorem:relaxedbehavior}N%Given I-state graph $I$, divulged information $D$, world graph $W$, and labelO%map $h$, the set of estimated world states for any set of I-states $B\subseteqH%V(I)$ is $\compatablew{I, D}{B}$. By replacing $D$ with its image graphB%$h\langle D\rangle$, the set of estimated world states for $B$ isD%$\compatablew{I, h\langle D\rangle}{B}$. $\forall B\subseteq V(I)$,H%$\compatablew{I, h\langle D\rangle}{B}\supseteq \compatablew{I, D}{B}$.%\end{theorem}%\begin{proof}4%%, and $\widetilde{\mathcal{W}}_{B}^{h(D)}\supseteq#%%\widetilde{\mathcal{W}}_{B}^{D}$.C%According to Lemma \ref{lemma:hinvhpgraph}, $\Language{D}\subseteqP%\Language{\inv{h}\of h\langle D\rangle}$. Thus $\forall B\subseteq V(I),\forallQ%w\subseteq V(W)$, we have $\inv{h}[\exactreachings{I}{B}] \cap \Language{D} \cap?%\reachings{W}{w} \subseteq \inv{h}[\exactreachings{I}{B}] \capC%\Language{\inv{h}\of h\langle D\rangle} \cap \reachings{W}{w}$. IfL%$\inv{h}[\exactreachings{I}{B}] \cap \Language{D} \cap \reachings{W}{w}\neqK%\emptyset$, then $\inv{h}[\exactreachings{I}{B}] \cap \Language{\inv{h}\ofH%h\langle D\rangle} \cap \reachings{W}{w}\neq \emptyset$. Thus, if $v\inK%\compatablew{I, D}{B}$, then $v\in \compatablew{I, h\langle D\rangle}{B}$.D%Hence, $\forall B\subseteq V(I)$, we have $\compatablew{I, h\langle.%D\rangle}{B}\supseteq \compatablew{I, D}{B}$.%\qed%\end{proof}%%K%An example to illustrate Theorem~\ref{theorem:relaxedbehavior} is shown inF%Figure~\ref{fig:relaxedbehavior}. Given $W$ and $I$, $\compatablew{I,P%D}{\{i_0\}}=\{w_1\}$, while $\compatablew{I, h\langle D\rangle}{\{i_0\}}=\{w_1,O%w_2\}$. Hence, $\compatablew{I, h\langle D\rangle}{B}\supseteq \compatablew{I,%D}{B}$.=%Note that $D$ is divulged in the preimage space.\footnote{WeN%assume this because it leads to a stronger adversary; if the plan informationH%is divulged in the image space some additional degradation of knowledgeD%occurs---results showing this appear in the supplementary material,#%Section~\ref{suppl:plan_divulge}.}GDefinition~\ref{def:compatibleworldstates} requires the substitution ofJthe second term in the intersection with $\Language{D}$.  When only $D$ isGgiven, the most precise inference replaces $\compatablew{I, P}{B}$ with$\compatablew{I,D}{B}$:*\begin{definition}[estimated world states] \label{def:estimatedworldstates}MGiven an I-state graph $I$, divulged plan p-graph $D$, world p-graph $W$, andOlabel map $h$, the set of estimated world states for I-states $B\subseteq V(I)$+is $\compatablew{I,D}{B} \defeq \left\{w\inMV(W)\;\middle|\;(\exactreachings{\inv{h}\langle I\rangle}{B}\cap \Language{D}.\cap \reachings{W}{w})\neq \emptyset\right\}$.\end{definition}\vspace*{-6pt}DObserve that $\inv{h}[\exactreachings{I}{B}]$ has been replaced with4$\exactreachings{\inv{h}\langle I\rangle}{B}$, sinceM$\inv{h}[\exactreachings{I}{B}]=\exactreachings{\inv{h}\langle I\rangle}{B}$.\label{sec:discussion_of_d}HThe last remaining element in Figure~\ref{fig:dataflow} that needs to beKaddressed is the stipulation of information. We do that next.\vspace*{-6pt}%\subsection{Information stipulations}J% <jmo>This section is missing something.  The propositions are defined inO% terms of I-states B.  Do we require that \Phi must hold for /every/ reachableN% B?  Or something else?  This shows up in the definition of Check in the next#% section, but that seems too late.% O% It seems cleanest to define here what it means for \Phi to "hold", and in the5% definition of Check, simply require the \Phi holds.%N% This is a big enough modification that I have not attempted it unilaterally.% </jmo>M% <das>I think that there is merit in doing this, but I've run out of time toM% pull that forward to here. As I looked I realized that there's a mixture ofL% details in that other section SDE, intersections of languages, etc. </das>PWe prescribe properties of the information that an observer may extract from itsIinput by imposing constraints on the sets of estimated world states.  TheKobserver, filtering a stream of inputs sequentially, forms a correspondenceLbetween its I-states and world states.  We write propositional formulas withAsemantics defined in terms of this correspondence---in this modelGthe stipulations are written to hold over \emph{every} reachable set ofOassociated states.\footnote{We foresee other variants which are straightforwardPto modifications to consider; but we report only on our current implementation.} \begin{figure}[t]{%\footnotesize\begin{equation*}\begin{aligned}M&{\rm Formula} \rightarrow {\rm Clause}_1 \form{\land} \dots\form{\land} {\rmNClause}_n\\[-2pt] &{\rm Clause}  \rightarrow {\rm Literal}_1 \form{\lor} \dots#\form{\lor} {\rm Literal}_m\\[-2pt]M&{\rm Literal} \rightarrow {\rm Symbol}\;|\; \form{\neg} {\rm Symbol}\\[-2pt]F&{\rm Symbol}  \rightarrow \form{\mathpzc{v_0}}, \form{\mathpzc{v_1}},#\form{\mathpzc{v_2}}, \dots\\[-8pt]
\end{aligned}\end{equation*}}\\!\setlength{\extrarowheight}{20pt}\setlength{\tabcolsep}{12pt}
\centering\begin{tabular}{cc}6\infer{ \langle \form{\mathpzc{v_i}} \rangle\Downarrow=\operatorname{eval}(v_i\stackrel{?}{\in} \compatablew{I,D}{B}N)}{[{\text{\footnotesize VALUE}}]~} & \infer{\langle \form{\neg \mathpzc{v_i}}P\rangle \Downarrow \text{the negation of }w}{[\textrm{\footnotesize NOT}]~~~~~~~N\langle \form{\mathpzc{v_i}}\rangle \Downarrow w~~~~~~~~~} \\ {\infer{ \langleJ\form{\mathpzc{\ell_1} \lor \mathpzc{\ell_2}} \rangle \Downarrow \text{theLlogical $\operatorname{or}$ of $w_1$ and $w_2$}}{[\textrm{\footnotesize OR}]K~~~~~~~~~ \langle \form{\mathpzc{\ell_1}} \rangle \Downarrow w_1~~~ \langleK\form{\mathpzc{\ell_2}}\rangle  \Downarrow w_2~~~~~~~~~}} & {\infer{\langleM\form{\mathpzc{c_1} \land \mathpzc{c_2}} \rangle \Downarrow \text{the logicalP$\operatorname{and}$ of $w_1$ and $w_2$}}{[\textrm{\footnotesize AND}] ~~~~~~~~~>\langle \form{\mathpzc{c_1}} \rangle \Downarrow w_1~~~ \langleC\form{\mathpzc{c_2}}\rangle \Downarrow w_2~~~~~~}} \\ \end{tabular}J\caption{The syntax and natural semantics of the information stipulations,Pwhere $\form{\mathpzc{c_i}}$, $\form{\mathpzc{\ell_i}}$, $\form{\mathpzc{v_i}}$,Nrepresent a clause, literal, and symbol, respectively, and $w_i$ is the resultNof the evaluation. The transition $\langle\form{\mathpzc{e}}\rangle \DownarrowPw$ denotes a transition, where $\form{\mathpzc{e}}$ is any expression defined by=the grammar and $w$ is the value yielded by the expression. }&\label{suppl:fig:operationalsemantics}\vspace*{-1pt}\end{figure}P%(Within the supplementary material, Figure~\ref{suppl:fig:operationalsemantics}%summarizes the semantics).K% depend on the evolution of the observer's I-states over time, rather thanH% only on instantaneous snapshots?  For example, perhaps it's OK for theK% observer to know that I'm in state X intermittently, but I don't want theP% observer to be able to infer that I've stayed in X for, say, three consecutiveN% time steps.  Possibly we can write LTL formulas that describe these kinds of% temporal constraints.</jmo>O% <yulin>: Yes. We've got a transition system for most of the problems (the oneL% involves finding label map may require extra efforts to build a transitionM% system). We can definitely specify stipulations with LTL or CTL, instead ofO% CNFs used in this paper. Furthermore, we can also allow the user to introduceO% additional properties associated with the transition system, then specify theH% stipulations on these properties. For example, "consecutive time steps% staying in X" is a property. K% More future work about stipulations: The stipulations can also serve as aN% knowledge for the robot or the observer to guide its estimation or planning.L% 1, we can convert the stipulations (propositional logic, LTL, CTL) to someL% p-graph or automata.  (a). Convert DNF to a p-graph, following the idea ofP% universal graph in discreet discrete paper.  (b). LTL, CTL can be converted toN% buchi automata or related automata.  2. use the p-graph or automata like theN% disclosed plan to guide estimation or planning.  P-graph seems natural to beO% integrated. But it may take extra efforts to think about how to use automata.
% </yulin>OFirst, however, we must delineate the scope of the estimated world states to beIconstrained.  Some states, in inherently non-deterministic worlds, may be?inseparable because they are reached by the same execution.  InKFigure~\ref{fig:stringbehindvertices}, both $w_3$ and $w_4$ will be reachedM(non-deterministically) by execution $a_2o_1$. Since this is intrinsic to theCworld, even when the observer has perfect observations, they remainJindistinguishable. In the remainder of this paper, we will assume that theMworld graph $W$ is in state-determined form, and we may affix stipulations toKthe world states knowing that no two vertices will be non-deterministicallyreached by the same execution.MSecond, we write propositional formulae to constrain the observer's estimate.JFormula $\form{\Phi}$ is written in conjunctive normal form, consisting ofPsymbols, literals and clauses as shown in \fref{suppl:fig:operationalsemantics}.NFirstly, an atomic symbol $\form{\mathpzc{v_i}}$ is associated with each worldFstate $v_i\in V(W)$. If $v_i$ is contained in the observer's estimatesB$\compatablew{I, D}{B}$, we will evaluate the corresponding symbolL$\form{\mathpzc{v_i}}$ as $\True$.  It evaluates as $\False$ otherwise. WithBeach symbol grounded in this way, we evaluate literals and clausesPcompositionally, using logic operators {\sc not}, {\sc and}, {\sc or}. These areOdefined in the standard way, eventually enabling evaluation of $\form{\Phi}$ on0the observer's estimate $\compatablew{I, D}{B}$.\gobble{Suppose, forJexample, we wish to require that state $v_1$ be included in the observer's%estimates of the world whenever $v_2$Dis; this would be expressed via \form{$\Phi = \neg\mathpzc{v_2} \veeP\mathpzc{v_1}$}.  Evaluation of such formulas takes place as follows.  For a setGof I-states $B$ reached under some operation of the robot in the world,M\form{$\mathpzc{v_i}$} is connected with $v_i$ in that \form{$\mathpzc{v_i}$}Devaluates to \True for $B$ iff $v_i \in \compatablew{I,D}{B}$, whereM$\compatablew{I,D}{B}$ is the set of estimated world states for I-states $B$.LThe opposite condition, where $v_i\not \in \compatablew{I,D}{B}$, is writtenNnaturally as \form{$\neg \mathpzc{v_i}$}.  Standard connectives \form{$\neg$},F\form{$\land$}, \form{$\lor$} enable composite expressions for complexstipulations to be built.}JLet the predicate $\operatorname{satfd}(B,\form{\Phi})$ denote whether theOstipulation $\form{\Phi}$ holds for I-states $B$. Then a plan $P$ satisfies thestipulations, if and only if
\vspace{-4pt}\[,\forall s\in \Language{P}\cap \Language{W}\;=B=\reachedv{I}{h(s)}\,\;\operatorname{satfd}(B,\form{\Phi}). \vspace*{-5pt}\]6\section{The observer's knowledge of the robot's plan}\label{section:divulgedplan}H% Somewhere in this section we need to define the most powerful observerPAbove, we hinted that observers may differ depending on the prior knowledge thatLhas been revealed to them;  next we bring this idea into sharper focus.  TheKinformation associated with an observer is contained in a pair $(I,D)$: thePI-state graph $I$ that acts as a filter, succinctly tracking state from a streamMof inputs, and knowledge of robot's plan in the form of a p-graph $D$.  TheseJtwo elements, through Definition~\ref{def:estimatedworldstates}, allow theKobserver to form a correspondence with the external world $W$.  The I-stateMgraph $I$ induces $\veq{}{I}{}$ over its set of executions and hence over theOjoint-executions with the world, or, more precisely, the image of those throughT$h$. \gobble{(Recall that the coarseness of $h$ limits the fidelity of the observer,Ksee Figure~\ref{suppl:fig:infocollapse}.)} By comparing the fineness of theLrelations induced by two I-state graphs, one obtains a sense of the relativeMcoarseness of the two I-state graphs.  As the present paper describes methodsGmotivated by applications to robotic privacy, we model the most capableLadversary, taking the \emph{finest observer}, that is, one whose equivalence!classes are as small as possible.\vspace*{-2pt}#\begin{definition}[finest observer]OGiven world graph $W$ and the divulged plan $D$, an I-state graph $\fine{I}$ isLa \emph{finest observer} if for any I-state graph $I$, we have $\forall s\inG\Language{W}$, $\compatablew{\fine{I},D}{h(s)}\subseteq \compatablew{I,
D}{h(s)}$.\end{definition}\vspace*{-2pt}M%Any observer attempting to keep more states that ${V(W)}$ is merely trackingN%repetitions and, given the form of the stipulations, this grants the observerL%no additional capabilities. Hence, the finest observer is well defined and %also conveniently represented:%,%\begin{lemma} Three closely related claims:%\label{lemma:two_cases}%\begin{itemize}C%\item [${}^{{}_\bullet}$]For every pair of executions $s_1, s_2\inQ%\inv{h}[\exactreachings{h\langle {W}\rangle}{B}]\cap \Language{W}$, then we haveA%either $h(s_1)=h(s_2)$ or $\reachedv{W}{s_1}=\reachedv{W}{s_2}$.%C%\item [${}^{{}_\bullet}$]For every pair of executions $s_1, s_2\inQ%\inv{h}[\exactreachings{h\langle {D}\rangle}{B}]\cap \Language{D}$, then we haveA%either $h(s_1)=h(s_2)$ or $\reachedv{D}{s_1}=\reachedv{D}{s_2}$.%C%\item [${}^{{}_\bullet}$]For every pair of executions $s_1, s_2\inO%\inv{h}[\exactreachings{h\langle {W}\pgprod D\rangle}{B}]\cap \Language{W}\capK%\Language{D}$, then we have either \underline{$h(s_1)=h(s_2)$}, or we have5%\underline{$\reachedv{W}{s_1}=\reachedv{W}{s_2}$ and(%$\reachedv{D}{s_1}=\reachedv{D}{s_2}$}.%%\end{itemize}%\end{lemma}%\begin{proof}M%For $\forall s_1, s_2\in \inv{h}[\exactreachings{h\langle W\rangle}{B}]$, we@%have $\reachedv{h\langle {W}\rangle}{h(s_1)}=\reachedv{h\langle8%{W}\rangle}{h(s_2)}=B$. Suppose $h(s_1)\neq h(s_2)$ andM%$\reachedv{W}{s_1}\neq\reachedv{W}{s_2}$. Let $w_1\in \reachedv{W}{s_1}$ andH%$w_1\not\in \reachedv{W}{s_2}$. Then we have $w_1\in \reachedv{h\langle>%{W}\rangle}{h(s_1)}$. In order to satisfy $\reachedv{h\langleM%{W}\rangle}{h(s_1)}=\reachedv{h\langle {W}\rangle}{h(s_2)}$, we need to findN%another execution $s'\in \inv{h}[\exactreachings{h\langle W\rangle}{B}]$ suchN%that $h(s')=h(s_2)$ and $w_1\in \reachedv{W}{s'}$, which contradicts with theI%condition $s_1, s_2\in \inv{h}[\exactreachings{h\langle W\rangle}{B}]$. %Q%Similarly, one proves that $\forall s_1, s_2\in \inv{h}[\exactreachings{h\langleJ%{D}\rangle}{B}]\cap \Language{D}$, then we have either $h(s_1)=h(s_2)$ or(%$\reachedv{D}{s_1}=\reachedv{D}{s_2}$. %F%Now, $\forall s_1, s_2\in \inv{h}[\exactreachings{h\langle {W}\pgprodG%D\rangle}{B}]\cap \Language{W}\cap \Language{D}$, we have $s_1, s_2\inN%\inv{h}[\exactreachings{h\langle {W}\rangle}{B'}]\cap \Language{W}$ and $s_1,M%s_2\in \inv{h}[\exactreachings{h\langle {D}\rangle}{B''}]\cap \Language{D}$.8%Then we have either $h(s_1)=h(s_2)$ or, otherwise, bothQ%$\reachedv{W}{s_1}=\reachedv{W}{s_2}$ and $\reachedv{D}{s_1}=\reachedv{D}{s_2}$.%\qed\end{proof}%TODO: change $D$ to $P$+\begin{restatable}[]{lemma}{finestobserver}\label{lemma:finestobserver})$h\langle W\rangle$ is a finest observer.\end{restatable}KThe observer only ever sees the image of the world under the label map $h$,Ni.e.  $h\langle W\rangle$. The p-graph $h\langle W\rangle$ serves as a naturalEI-state graph for a finest observer as it allows the observer to haveAsufficient internal structure to keep track of every world state.%\begin{proof}G%This lemma will be proved by showing that $\forall s\in \Language{W}$,L%$B=\reachedv{h\langle W\rangle}{h(s)}$ and $B'=\reachedv{I}{h(s)}$, we haveI%$\compatablew{h\langle W\rangle, D}{B}\subseteq \compatablew{I, D}{B'}$.%N%If $\compatablew{h\langle W\rangle, D}{B}$ contains only one world state $w$,F%then $\exists s\in \inv{h}[\exactreachings{h\langle W\rangle}{B}]\cap6%\Language{D}\cap\reachings{W}{w}$. We also have $s\inO%\inv{h}[\exactreachings{I}{B'}]\cap \Language{D}\cap\reachings{W}{w}$ for someN%$B'\subseteq V(I)$. Hence, $w$ is also contained in $\compatablew{I, D}{B'}$.%M%If $\compatablew{h\langle W\rangle, D}{B}$ contains at least two world stateP%$w_1, w_2$, then $\exists s_1\in {h}[\exactreachings{h\langle W\rangle}{B}]\capM%\Language{D}\cap\reachings{W}{w_1}$ and $s_2\in {h}[\exactreachings{h\langleD%W\rangle}{B}]\cap \Language{D}\cap\reachings{W}{w_2}$. According to0%Lemma~\ref{lemma:two_cases}, $h(s_1)=h(s_2)$ orN%$\reachedv{W}{s_1}=\reachedv{W}{s_2}$. If $h(s_1)=h(s_2)$, then it is trivialH%that $s_1,s_2\in \inv{h}[\exactreachings{I}{B'}]$. We have $s_1\in {h}[L%\exactreachings{I}{B'}]\cap \Language{D}\cap\reachings{W}{w_1}$ and $s_2\inL%{h}[\exactreachings{I}{B'}]\cap \Language{D}\cap\reachings{W}{w_2}$. Hence,P%$w_1, w_2\in \compatablew{I, D}{B'}$. If $\reachedv{W}{s_1}=\reachedv{W}{s_2}$,2%then $s_1, s_2\in {h}[\exactreachings{I}{B'}]\capF%\Language{D}\cap\exactreachings{W}{\{w_1,w_2\}}$. Hence, $w_1, w_2\in"%\compatablew{I}{B'}$ also holds. %K%Therefore, $\compatablew{h\langle W\rangle, D}{B}\subseteq \compatablew{I,K%D}{B'}$. Hance, $h\langle W\rangle$ is a finest observer. \qed \end{proof}%\begin{proof}9%The proof appears in the supplementary material, Section,%\ref{suppl:obs_lemmas_proofs}.  \end{proof}OThe second element in the observer pair is $D$, information disclosed about theNplan, and presumed to be known \emph{a priori}, to the observer.  Depending onNhow much the observer knows, there are multiple possibilities here, from most-to least-informed:\begin{enumerate}(\renewcommand{\theenumi}{\Roman{enumi}}%1\item The observer knows the exact plan $P$ to beOexecuted.\label{item:exactplan} \item The plan to be executed is among a finiteBcollection of plans $\{P_1,P_2, \dots, P_n\}$.\label{item:setplan}P%\item Furthermore, the plan to be executed can be made from the segments of theD%following plans $\{P_1,P_2, \dots, P_n\}$, where $n$ is an integer.O%Specifically, the robot can choose the actions from multiple plans, as long asN%it is safe on the world graph $W$ and solves the planning problem. An exampleO%to construct a new plan $P_3$ from the segments of $P_1$ and $P_2$ is shown inP%Figure \ref{fig:closure}, where $P_3$ can generate some execution that does notP%belong to either $P_1$ or $P_2$.  To distinguish it from previous case, we willK%put an closure operator $\circlesign{*} W$ on the set: $\{P_1, P_2, \dots,O%P_n\}^{\circlesign{*}W}$, where $W$ indicates that the closure operator on theM%plan segments should be safe on the world graph $W$; \todo{Additional bulletP%point: combine the set of plans with plan closure. Note that $P^{\star}$ is the%plan closure.}N\item The observer may only know that the robot is executing \emph{some} plan,Ethat is, the robot is goal directed and aims to achieve some state inM$V_{\goal}$. \label{item:someplan} \item The observer knows nothing about theKrobot's execution other than that it is on $W$. \label{item:wanderingrobot}\end{enumerate}%\vspace*{-1pt}NIt turns out that a p-graph exists whose language expresses knowledge for each7of those cases (we omit the details here). Furthermore,MSection~\ref{sec:discussion_of_d} details how the observer's knowledge of theAworld state ($\compatablew{I,D}{B}$) from I-states $B$ depends onP$\exactreachings{\inv{h}\langle I\rangle}{B}\cap \Language{D}\cap \Language{W}$,Aa set of executions that arrive at $B$ in the I-state graph $I$. GBecause the observer uses $D$ to refine $\exactreachings{\inv{h}\langleNI\rangle}{B}$, when $\Language{P} \subsetneq \Language{D}$ the gap between theLtwo sets of executions represents a form of uncertainty. The ordering of theIfour cases, thus, can be stated precisely in terms of language inclusion.ONow using the $D$ as appropriate for each case, one may examine whether a givenNplan and disclosure policy solves the planning problem (i.e., achieves desiredOgoals in the world) while meeting the stipulations on information communicated.PHence, we see that describing disclosed information via a p-graph $D$ is in factOrather expressive. This section has also illustrated the benefits of being ableIto use both interaction language and graph presentation views of the same
structure.\vspace*{-6pt}L\section{Searching for plans and disclosure policy: the {\sc Seek} problems}\label{section:search}OIn this section, we will show how to search for a plan (together with the labelmap).P%Naturally, the question of most interest is how to find plans and/or disclosureM%policies, not merely how to verify them. We formulate this search problem in%three varieties: \par\medskip
    \noindent
    \fbox{(        \begin{minipage}{0.96\textwidth}:               \noindent\begin{minipage}[t]{0.6\textwidth}S                        \textbf{Problem:} {\nameseekp$\big((W, V_{\goal}), \var{x},H                            (\fine{I}, D), h, \form{\Phi}\big)$\newline &                        \hspace*{40pt}W                        \nameseekplm$\big((W, V_{\goal}), \var{x}, (\fine{I}, \var{x}),D                        \var{\lambda}, \form{\Phi}\big)$} \\[0.05in]#               \end{minipage}      \hfill:               \noindent\begin{minipage}[t]{0.3\textwidth}               \fbox{6                   \begin{minipage}[t]{0.75\textwidth}N                            {\footnotesize Vars. to solve for:\vspace*{-2pt}\\H                            \null~~\,$\var{x}$ is a plan\vspace*{-2pt}\\C                            \null~~~$\var{\lambda}$ is a label map}"                    \end{minipage}               }               \end{minipage}           7               \noindent\begin{minipage}[t]{\textwidth}%                    %\vspace*{-0.4cm}2                    \renewcommand{\tabcolsep}{2pt}4                    \begin{tabularx}{\linewidth}{rX}N                        \emph{Input:} & A planning problem $(W, V_{\goal})$, aP                        finest observer $\fine{I}$, a divulged plan p-graph $D$,U                        information disclosure policy $h$ and information stipulation(                        $\form{\Phi}.$\\S                        \emph{Output:} & A plan ${\var{x} = (P, V_{\term})}$ and/orU                        label map ${\var{\lambda}=h}$ such that plan $(P, V_{\term})$N                        solves the problem $(W, V_{\goal})$, and $\forall s\inV                        \Language{W^{\dagger}}\cap\Language{P}, B=\reachedv{I}{h(s)}$,X                        the information stipulation $\form{\Phi}$ is always evaluated as=                        \True on $\compatablew{I,D}{B}$ (i.e.R                        $\operatorname{satfd}(B,\form{\Phi})=\True$), else \False."                    \end{tabularx}                \end{minipage}                         \end{minipage}     }	%\medskip%\par
%\ourproblem{?%    \nameseekp$\big((W, V_{\goal}), \var{x}, (\fine{I}, D), h,+%\form{\Phi}\big)$\newline \hspace*{43.5pt}S%%    \nameseeklm$\big((W, V_{\goal}), (P, V_{\term}), (\fine{I}, D), \var{\lambda}-%%,\form{\Phi}\big)$\newline \hspace*{43.5pt}D%    \nameseekplm$\big((W, V_{\goal}), \var{x}, (\fine{I}, \var{x}),#%\var{\lambda}, \form{\Phi}\big)$ }%	{Data as before, /%    %but here $I$ is any given observer while *%    but $\fine{I}$ is a finest observer.}L%    %, $\var{x}$ as the plan to be searched for (and disclosed in the third?%    %case),  $\var{\lambda}$ as the label map to be searched.}N%    {A plan ${\var{x} = (P, V_{\term})}$ and/or label map ${\var{\lambda}=h}$M%    such that plan $(P, V_{\term})$ solves the problem $(W, V_{\goal})$, andQ%    $\forall s\in \Language{W^{\dagger}}\cap\Language{P}, B=\reachedv{I}{h(s)}$,N%    the information stipulation $\form{\Phi}$ is always evaluated as \True onO%    $\compatablew{I,D}{B}$ (i.e. $\operatorname{satfd}(B,\form{\Phi})=\True$),%    else \False.} %\vspace*{-3.6cm}%\hspace*{8.95cm}\fbox{%\begin{minipage}[t]{2.2cm}3%{\footnotesize Vars. to solve for:\vspace*{-2pt}\\-%\null~~\,$\var{x}$ is a plan\vspace*{-2pt}\\(%\null~~~$\var{\lambda}$ is a label map}%\end{minipage}%}LOf the two versions of {\sc Seek}, the first searches for a plan, the second?for a plan and a label map, jointly.  We consider each in turn.C%But it is not merely the joint search that makes the third problemL%substantially more challenging and also interesting.  Whereas the first twoG%have a divulged plan $D$ that is \emph{a priori} fixed, the third usesC%$\var{x}$, the plan that was found, as $D$. This latter fact makes;%the third substantially more difficult than the other two.%P%At a high-level, it is not hard to see why: the definitions in the previous twoN%sections show that both $P$ and $D$ play a role in determining whether a planP%satisfies a stipulation.  Where $D$ is known and fixed beforehand (for example,G%in Case~\ref{item:wanderingrobot}, $D=W$, or Case~\ref{item:someplan},N%$D=P^{*}$), a solution can proceed by building a correspondence in the tripleP%graph ${W\pgprod D\pgprod \inv{h}\langle \sde{I}\rangle}$ and searching in thisN%graph for a plan.  In \nameseekplm\xspace,  however, one is interested in theM%case where ${D=P}$, where the divulged plan is tight, being the robot's planP%exactly.  We cannot search in the same product graph, because we can't make theO%correspondence since $D$ has yet to be discovered, being determined only afterP%$P$ has been found.  Crucially, the feasibility of $P$ depends on $D$, that is,M%on itself!  Finding such a solution requires an approach capable of buildingN%incremental correspondences from partial plans.  The key result of this paperK%is that \nameseekplm\xspace is actually solvable without resorting to mere%generate-and-{\sc Check}.%O%Though the algorithm for \nameseekplm\xspace can also solve for only plans, orI%only label maps, and can be modified to match some $D$, the algorithm weL%describe has double exponential cost. Consequently, it is worth consideringN%specialized approaches for \nameseekp\xspace and \nameseeklm\xspace, if those%are the instances one faces.N%not merely verifying their suitability. We firstly consider seeking plans forO%given world, divulged plan and an arbitrary observer, where the correspondenceP%is established in the triple graph of the three and the plan can be searched inO%this triple graph.  Instead of considering all the possible plans, which couldE%be enormous, we narrow the search space down into a kernel with onlyO%$c$-bounded congruent plan. \textcolor{blue}{In the above problem, we are ableP%to conduct our search on graph which established the correspondence between theL%estimated world states or plan states to the observer I-state.} But this isO%inapplicable for seeking a plan and label map, when the plan is also disclosedL%to the `finest' observer, which is the best filter the observer can get. InJ%this problem, the divulged plan as a key part of the triple graph, is theK%element to be searched. There is no way for us to build the correspondenceM%before the search, which drastically increases the computation complexity toM%search for such a solution. We will demonstrate an algorithm which is double#%exponential to solve this problem.<%TODO: We move seek label map to the supplementary material.\vspace*{-5pt}   8\subsection{Finding a plan given some predetermined $D$}\label{section:seekplan}OFor \nameseekp\xspace, first we must consider the search space of plans.  PriorNwork~\cite{saberifar18pgraph} showed that, although planning problems can haveFstranger solutions than people usually contemplate, there is a core ofNwell-structured plans (called homomorphic solutions) that suffice to determineGsolvability. As an example, there may exist plans which loop around theNenvironment before achieving the goal, but, they showed that in seeking plans,;one need only consider plans that short-circuit the loops. NThe situation is rather different when a plan must satisfy more than mere goalOachievement: information stipulations may actually {\sl require} a plan to loopMin order to ensure that the disclosed stream of events is appropriate for theMobserver's eyes. (A concrete example appears in \fref{fig:exp-nuclear}(c).)  %<jmo>Wait, where?</jmo>OThe argument in~\cite{saberifar18pgraph} needs modification for our problem---aMdifferent construction can save the result even under disclosure constraints.4This fact is key to be able to implement a solution.=%Since the space of all p-graphs is extremely large, we must P%narrow down the space in which we search for $P$. The following definitions and%lemmas help in this regard.GIn this paper, without loss of generality, we focus on finding plans inIstate-determined form.  Next, we will examine the solution space closely.\begin{definition}\label{def:congruent}OA plan $P$ is \emph{congruent} on the world graph $W$, if and only if for everyHpair of executions $s_1, s_2\in \Language{P}$ we have $\veq{s_1}{P}{s_2}\implies \veq{s_1}{W}{s_2}$.\end{definition}\vspace*{-5pt}OIn other words, a plan that respects the equivalence classes of the world graphKis defined as a congruent plan.  Next, our search space is narrowed furtherstill.]%Next, we going to narrow our search space to the set of congruent plans with bounded length.
\begin{lemma}\label{lemma:congruentplan}P%Given any plan $P$ that is a solution of problem $P1$, then there exists a planO%$P'$ that is both a solution of problem $P1$, and congruent on the world graph%$W$. OGiven any plan $(P, V_{\term})$, there exists a plan $(P', V'_{\term})$ that isBcongruent on the world graph $W$ and $\Language{P'}=\Language{P}$.%\end{restatable}\end{lemma}
\begin{proof}MWe give a construction from $P$ of $P'$ as a tree, and show that it meets theconditions.@%that $P'$ is congruent on $W$ and $\Language{P'}=\Language{P}$.HTo construct $P'$, perform a BFS on $P$. Starting from $V_0(P)$, build aMstarting vertex $v_0$ in $P'$, keep a correspondence between it and $V_0(P)$.PMark $v_0$ as unexpanded. Now, for every unexpanded vertex $v$ in $P'$, mark theJset of all outgoing labels for its corresponding vertices in $P$ as $L_v$,Ncreate a new vertex $v'$ in $P'$ for each label $l\in L_v$, build an edge fromP$v$ to $v'$ with label $l$ in $P'$, and mark it as expanded. Repeat this processNuntil all vertices in $P'$ have been expanded. Mark the vertices correspondingOto vertices in $V_{\term}$ as $V'_{\term}$. In the new plan $(P', V'_{\term})$,Fno two executions reach the same vertex. That is, $\forall s_1, s_2\inI\Language{P'}, \nveq{s_1}{P'}{s_2}$. Hence, $P'$ is congruent on $W$.  InMaddition, since no new executions are introduced and no executions in $P$ are3eliminated during the construction of $P'$, we have-$\Language{P'}=\Language{P}$. \qed\end{proof}-\begin{restatable}[]{theorem}{boundcongruent}\label{thm:boundcongruent}UFor problem \nameseekp$\big((W, V_{\goal}), \var{x}, (I, D), h, \form{\Phi}\big)$, ifLthere exists a solution $(P, V_{\term})$, then there exists a solution $(P',KV'_{\term})$ that is both $c$-bounded and congruent on $W$, where $c=|V(W)|\cdot |V(D)|\cdot |V(I)|$.\end{restatable}\vspace*{-10pt}
\begin{proof}NSuppose \nameseekp\xspace has a solution $(P, V_{\term})$.  Then the existenceHof a solution $(P', V'_{\term})$ which is congruent on $W$ is implied by3Lemma~\ref{lemma:congruentplan}. Moreover, we have J{\sc Check}$\big((W, V_{\goal}), (P, V_{\term}), D, I, h, \form{\Phi}\big)	\implies$E{\sc Check}$\big((W, V_{\goal}),$ $(P', V'_{\term}),$ $D,$ $ I,$ $ h,\form{\Phi}\big)$, following from two observations:\begin{tightenumerate}O\item[(\textit{i}.)] if $(P, V_{\term})$ solves $(W, V_{\goal})$ then the means<of construction ensures $(P', V'_{\term})$ does as well, andK\item[(\textit{ii}.)] in checking $\form{\Phi}$, the set of estimated worldGstates $\compatablew{I,D}{\{v\}}$ does not change for each vertex $v\inNV(\sde{I})$, since the triple graph is independent of the plan to be searched.DThe set of I-states to be evaluated by $\form{\Phi}$ in $\sde{I}$ isM$\cup_{s'\in h[\Language{P}\cap \Language{W}]}\reachedv{\sde{I}}{s'}$.  SinceO$\Language{P}=\Language{P'}$, the set of I-states to be evaluated is no altered;and the truth of $\form{\Phi}$ along the plan is preserved.\end{tightenumerate}
\vspace*{4pt}N\noindent The final step is to prove that if there exists a congruent solutionK$(P', V'_{\term})$, then there exits a solution $(P'', V''_{term})$ that isN$c$-bounded. First, build a product graph $T$ of $W$, $D$, and $\inv{h}\langleGSED(I)\rangle$, with vertex set $V(W)\times V(D)\times V(\inv{h}\langleN\sde{I}\rangle)$. Then trace every execution $s$ in $P'$ on $T$. If $s$ visitsJthe same vertex $(v^W, v^{D}, v^{\inv{h}\langle \sde{I}\rangle})$ multipleLtimes, then $v^W$, $v^D$, and $v^{\inv{h}\langle \sde{I}\rangle}$ have to beJaction vertices, for otherwise $P'$ can loop forever and is not a solutionO(since $P'$ is finite on $W$).  Next, record the action taken at the last visitOof $(v^W, v^P, v^{\inv{h}\langle \sde{I}\rangle})$ as $a_{\rm last}$.  Finally,Lbuild a new plan $(P'', V'_{\term})$ by bypassing unnecessary transitions onP$P'$ as follows. For each vertex $(v^W, v^P, v^{\inv{h}\langle \sde{I}\rangle})$Mthat is visited multiple times, $P''$ takes action $a_{\rm last}$ when $(v^W,Ov^P, v^{\inv{h}\langle \sde{I}\rangle})$ is first visited.  $P''$ terminates atMthe goal states without violating any stipulations, since it takes a shortcutMin the executions of $P'$ but---crucially---without visiting any new observerPI-states. In addition, $P''$ will visit each vertex in $T$ at most once, and the?maximum length of its executions is $|V(W)|\times |V(D)| \timesL|V(\inv{h}\langle \sde{I}\rangle)|$.  Since $P''$ preserves the structure of8$P'$ during this construction, $P''$ is also congruent. \qed\end{proof}%\begin{proof}1%The proof appears in the supplementary material,)%Section~\ref{suppl:plan_lemmas_proofs}. %\end{proof}LThe intuition, and the underlying reason for considering congruent plans, isJthat modifying the plan will not affect the stipulations if the underlyingJlanguages are preserved.  The bound on the length then takes this further,Imodifying the language by truncating long executions in the triple graph,Lthereby shortcutting visits to I-states that do not affect goal achievement.PAccordingly, it suffices to look for congruent plans in the (very specific) formOof trees, since any plan has a counterpart that is congruent and in the form of8a tree (see Lemma~\ref{lemma:congruentplan} for detail).MTheorem~\ref{thm:boundcongruent} states that the depth of the tree is at most@$c=|V(W)|\cdot |V(D)| \cdot |V(\inv{h}\langle \sde{I}\rangle)|$.OTherefore, we can limit the search space to trees of a specific bounded depth. LTo search for a $c$-bounded solution, first we mark the vertex $(v^W, v^{D},Fv^{\inv{h}\langle \sde{I}\rangle})$ as: (i)~a goal state if $v^{W}$ is a goal state in the world graph;L%all the world states appearing together with $v^{D}$ are goal states in the%world graph; P% <zyl> There is no need for the robot to be ignorance by merging some states inK% the plan, since the observer does not know the robot's exact plan but $D$K% instead. Hence, we can always assume that the robot knows the exact world(% state, given a state-determined world.% </zyl>M(ii)~as satisfying $\form{\Phi}$ when all the world states appearing togetherNwith $v^{\inv{h}\langle \sde{I}\rangle}$ together satisfy $\form{\Phi}$.  ThenJwe will conduct an \aNd--\Or search~\cite{pearl84heuristics} on the triplegraph:\vspace*{-4pt}\begin{itemize}O\item [$\bullet$]Each action vertex serves as an \Or node, and an action shouldMbe chosen for the action vertex such that it will eventually terminate at theEgoal states and all the vertices satisfy $\form{\Phi}$ along the way.N\item [$\bullet$]Each observation vertex is treated as an \aNd node, and thereKexists a plan that satisfies $\form{\Phi}$ for all its outgoing observation	vertices.
\end{itemize}9%A detailed algorithm is given in the supplementary file.K%The astute reader will have observed that the $\fine{I}$ could just be anyG% $I$; no special assumptions are made about it. That is true: all the :% preceding results hold for any observer's I-state graph.\vspace*{-8pt}M\subsection{Search for plan and label map for the finest observer, disclosing
the same}  \label{section:seekplanlabelmap}JIt is not merely the joint search that makes this, the second problem moreFinteresting. Whereas the first has a divulged plan $D$ that is \emph{aOpriori} fixed, the second uses $\var{x}$, the plan that was found, as $D$. This9latter fact makes the third substantially more difficult.KAt a high level, it is not hard to see why: the definitions in the previousLsection show that both $P$ and $D$ play a role in determining whether a planOsatisfies a stipulation.  Where $D$ is known and fixed beforehand (for example,Fin Case~\ref{item:wanderingrobot}, $D=W$, or Case~\ref{item:someplan},M$D=P^{*}$), a solution can proceed by building a correspondence in the tripleOgraph ${W\pgprod D\pgprod \inv{h}\langle \sde{I}\rangle}$ and searching in thisMgraph for a plan.  In \nameseekplm\xspace,  however, one is interested in theLcase where ${D=P}$, where the divulged plan is tight, being the robot's planOexactly.  We cannot search in the same product graph, because we can't make theNcorrespondence since $D$ has yet to be discovered, being determined only afterN$P$ has been found. Crucially, the feasibility of $P$ depends on $D$, that is,Kon itself! Finding such a solution requires an approach capable of buildingNincremental correspondences from partial plans.  A key result of this paper isGthat \nameseekplm\xspace is actually solvable without resorting to meregenerate-and-check.O%Though the algorithm for \nameseekplm\xspace can also solve for only plans, orI%only label maps, and can be modified to match some $D$, the algorithm weL%describe has double exponential cost. Consequently, it is worth consideringN%specialized approaches for \nameseekp\xspace and \nameseeklm\xspace, if those%are the instances one faces.P%In this subsection, we are focusing on problem \nameseekplm, where the divulgedO%plan is exactly the plan to be executed. Both the plan and label map should be%searched for. 
\begin{lemma}\label{lemma:plan_state}HLet $\compatablew{}{}$ be estimated world states for the finest observerN$h\langle W\rangle$, and let~$w$ be the world state which is observable to theHrobot.  If there exists a solution for \nameseekplm, then there exists aIsolution that only visits each pair $(w, \compatablew{}{})$ at most once.\end{lemma}
\begin{proof}PLet $(P, V_{\term})$ and $h$ be a solution for \nameseekplm. Suppose $P$ visitedP$(w, \compatablew{}{})$ n times. Let the set of actions taken at $i$-th visit beL$A_i$. Then we can construct a new plan $(P', V_{\term})$ which always takesP$A_n$ at $(w, \compatablew{}{})$. If $P$ does not violate the stipulations, thenH$P'$ will never do since $P'$ is a shortcut of $P$ and never visits morePI-states than $P$ does. In addition, $P'$ will also terminate at the goal regionif $P$ does.  \qed\end{proof}K%Lemma~\ref{lemma:plan_state} enables us to search the plan with plan stateO%represented by a tuple $(w, \compatablew{h\langle W\rangle, P}{B})$, where $w$F%is the world state and $\compatablew{h\langle W\rangle, P}{B}$ is theF%observer's estimated world states for I-states $B\subseteq V(h\langle%W\rangle)$.P%Note that $\mathcal{V}^{W}_{s}\subseteq \mathcal{V}^{h\langle W\rangle}_{h(s)}$ %and $|\mathcal{V}^{W}_{s}|=1$. -\begin{restatable}[]{theorem}{planstatescope}OIf there exists a solution for \nameseekplm$\big((W, V_{\goal}), \var{x}, (I_f,M\var{x}), \var{\lambda}, \form{\Phi}\big)$. then there exists a plan $P$ thatOtakes $(w, \compatablew{h\langle W\rangle, P}{B})$ as its plan state, where $w$Ois the world state and the set $\compatablew{h\langle W\rangle, P}{B}$ consistsDof the estimated world states for I-states $B$. Furthermore, if $(w,D\compatablew{h\langle W\rangle, P}{B})\in V(P)$, then $\forall w'\inM\compatablew{h\langle W\rangle, P}{B}$, $(w', \compatablew{h\langle W\rangle,P}{B})\in V(P)$. \end{restatable}\vspace*{-8pt}%\begin{proof}1%The proof appears in the supplementary material,5%Section~\ref{suppl:plan_lemmas_proofs}.  \end{proof}%\vspace*{-5pt}
\begin{proof}OLemma~\ref{lemma:plan_state} shows that we can treat $(w, \compatablew{h\langleDW\rangle, P}{B})$ as the plan state for the plan to be searched for.JSince $w'\in \compatablew{h\langle W\rangle, P}{B}$, we have $\exists s\inF\reachings{W}{w'}\cap \Language{P}\cap\inv{h}[\exactreachings{h\langleMW\rangle}{B}]$. Since $s\in \Language{P}$, $s$ reaches $w$ and $h(s)$ reachesH$B$, we have $s$ reaches the tuple $(w', \compatablew{h\langle W\rangle,JP}{B})$. Hence, $(w', \compatablew{h\langle W\rangle, P}{B})\in V(P)$.\qed\end{proof}C%In problem {\sc Seek}$\big((W, V_{\goal}), ?, (I_f, ?), \text{?`},P%\form{\Phi}\big)$, the only gap between the observer and the robot is the label!%map. For each $w'\in \mathcal{}$%O%In this problem, the only gap between the observer and robot is the label map.L%Each state $w$ from estimated world states $\mathcal{W}^{h\langle W\rangle,G%P}_{B}$ is a possible world state under some realization of the plan. I\noindent In searching for $(P, V_{\term})$, for any action state ${v_p =;(w,\compatablew{h\langle W\rangle, P}{B})}$, we determine: \begin{description}O\item [\quad$w\in V_{\goal}:$] We must decide whether $v_p \in V_{\term}$ holdsor not;P\item [\quad$w\not\in V_{\goal}:$] We must choose the set of nonempty actions toPbe taken at $v_p$. It has to be a set of actions, since these chosen actions areLnot only aiming for the goal but also obfuscating each other under the labelmap. S%\changed{We will show this with the following example: In the world graph shown inL%\fref{suppl:fig:seekplm_actions}, we can just pick either $a_1$ or $a_2$ atF%$w_1$ in the planning problem. But to solve problem \nameseekplm withN%stipulation $\Phi=(\neg w_3 |w_4)\land(w_3|\neg w_4)$, we have to choose bothN%action $a_1$ and $a_2$ when reaching $w_1$, and map them to the same image inG%the label map. Then the observer will never be able to distinguish theJ%transitions to $w_3$ and $w_4$. Hence, it is necessary to choose a set ofK%actions at a particular world state in \nameseekplm, when the plan is also
%disclosed.} %\begin{figure}%\centering8%\includegraphics[scale=0.7]{figure/seekplm_actions.pdf}A%\caption{An example to show that actions should be obfuscated in1%\nameseekplm.\label{suppl:fig:seekplm_actions}} 
%\end{figure} >%(See Section~\ref{suppl:seekplm_actions} in the supplement.) \vspace*{-3pt}\end{description}\vspace*{-5pt}KA state ${v_p=(w, \compatablew{h\langle W\rangle, P}{B})}$ is a terminatingFstate in the plan when $\compatablew{h\langle W\rangle, P}{B}\subseteqV_{\goal}$.LWith action choices for each plan state $(w, \compatablew{h\langle W\rangle,OP}{B})$ and label map $h$, we are able to maintain transitions of the estimatedAworld states for $B'$ after observing the image~$x$. Now, if $(w,J\compatablew{h\langle W\rangle, P}{B})$ is an action state, let the set ofOactions taken at $w$ be $A_w$. Then the label map $h$ partitions the actions inL$\cup_{w\in \compatablew{h\langle W\rangle, P}{B}} A_w$ into groups, each ofOwhich shares the same image. The estimated worlds states for $B'$ transition interms of groups \vspace*{-6pt}{\small\begin{equation*}
\begin{split}D\compatablew{h\langle W\rangle, P}{B'}=\left\{w'\in V(W)\middle| (w,E\compatablew{h\langle W\rangle, P}{B}\right.)&\not\in V_{\term}, w\inO\compatablew{h\langle W\rangle, P}{B},\\[-7pt] &\left.\exists a\in A_w, h(a)=x,\trto{W}{w}{a}{w'}\right\}.\end{split}\end{equation*}\vspace*{-6pt}}MConversely, if $(w, \compatablew{h\langle W\rangle, P}{B})$ is an observationOstate, let the observations available at $w$ be $O_w$. Then $h$ also partitionsLthe observations in $\cup_{(w, \compatablew{h\langle W\rangle, P}{B})\not\inAV_{\term}} O_w$ and estimated world states for $B'$ transition as\vspace*{-2pt}{\small\begin{equation*}
\begin{split}E \compatablew{h\langle W\rangle, P}{B'}=\left\{w'\in V(W)\middle| (w,F \compatablew{h\langle W\rangle, P}{B}\right.)&\not\in V_{\term}, w\inI \compatablew{h\langle W\rangle, P}{B},\\[-7pt] &\left.\exists o'\in O_w,% h(o')=x, \trto{W}{w}{o}{w'}\right\}.\end{split}\end{equation*}\vspace*{-14pt}}FInstead of searching for the label map over the set of all actions andPobservations in $W$, we will first seek a partial label map for all observationsNor chosen actions for world states in $\compatablew{h\langle W\rangle, P}{B}$,Nand then incrementally consolidate them. Each partial label map is a partitionMof the events, making it easy to check whether two partial maps conflict whenthey are consolidated. N%If two events previously in two separate partitions now should be in the sameN%partition or vice visa, then there is a conflict. If there is no conflict andP%all the resulting I-states satisfy the stipulations, then we will commit to the=%partial label maps and keep searching for the next I-states.PIf two partial partitions disagree on a value, we backtrack in the search to tryDanother partition label map.  Putting it all together as detailed inF\fref{fig:searchtree}, we can build a type of \aNd--\Or search tree toincorporate these choices.%\vspace*{-16pt}\begin{SCfigure}
\centeringB\includegraphics[scale=0.28]{figure/searchtree.pdf}\vspace*{-10pt}S\caption{Solving the \nameseekplm\xspace problem via generalized \aNd--\Or search.%\vspace{2pt}\newline\small%{K\null\quad For a set of actions comprising a vertex $\compatablew{0}{}$ twoFtiers of \Or nodes are generated. The first is over subsets $(A^{0}_1,PA^{0}_2,\dots, A^{0}_m)$, being possible actions to the take; the second choosesHspecific partitions of values $\mathbb{P}_i=\{X_1, X_2, \dots\}$, (i.e.,Mpartial label maps).  A given partition is expanded as an \aNd node with eachPoutgoing edge bearing a group of events sharing the same image under the partiallabel map. \newlineM\null\quad Observation vertices $\compatablew{1}{}$ are expanded in a similarPway, but are simpler since we forgo the step involving choosing actions.\newline}M%\null\quad Each new vertex ${\mathcal{W'}\subseteq V_{\goal}}$, we mark as a%goal state.  \vspace*{10pt}%\label{fig:searchtree}%\vspace*{-12pt}%}\end{SCfigure}%\vspace*{-10pt}D% The following has been mentioned in the problem of searching plan.%\begin{itemize}K%\item For each AND node, all the branches should eventually reach the goalO%states and no stipulations are violated.  \item For each OR node, there existsM%one branch where the goal states will be eventually reached and stipulations%are satisfied along the way.%\end{itemize}J%Starting with $\compatablew{0}{}$ in the tree, if $\compatablew{0}{}$ areM%action vertices in $W$, then we will expand it as an \Or node.  The outgoingM%edges are all possible values for $(A^{0}_1, A^{0}_2,\dots, A^{0}_m)$, whereP%$A^{0}_i$ is the set of actions taken at $w^{0}_i$.  Specifically, if we decideO%to terminate at $w^{0}_i$, then $A^{0}_i=\emptyset$. For a given action choiceP%$(A^{0}_1, A^{0}_2,\dots, A^{0}_m)$, we will expand it as an \Or node to searchN%for partial label maps for the chosen actions. Each outgoing edge serves as aK%specific partition of values.  Given a partition $\mathbb{P}_i=\{X_1, X_2,P%\dots\}$, we will expand it as an \aNd node, where each outgoing edge bearing aN%group of events $X_j\in \mathbb{P}_i$ shares the same image under the partialL%label map.  For each new estimate $\mathcal{W'}$, if $\mathcal{W'}\subseteqI%V_{\goal}$, we mark it as a goal state. If $\mathcal{W'}$ has never beenI%visited before, then we add it to queue for later expansion. ObservationL%vertices $\compatablew{1}{}$ are expanded in a similar way, but without the%choice of actions. LIf there exists a plan and label map then for each $\compatablew{}{}$ in theNtree, there exists an action choice under which there exists a safe partition,7such that there exists a plan for all of its children.  \gobble{The pesudocode to searchJthe plan by expanding the action node and observation node in the \aNd-\OrOsearch tree is shown in Algorithm~\ref{alg:actexpand} and \ref{alg:obsexpand}.}L% Backtrack can be triggered when either a set of estimated world states areO% visited more than once or the partial label map conflicts with previous labelD% maps. If $\compatablew{}{}\subseteq V_{\goal}$, then we mark it inO% $V_{\term}$. The process terminates when a plan is found and no conflicts are#% found in the partial label maps. % B%(Detailed pseudocode appears in Algorithm~\ref{alg:actexpand} and(%\ref{alg:obsexpand} of the supplement.)%\begin{algorithm}>%\caption{ActionExpand($\compatablew{}{}, P, h, \form{\Phi})$}%\label{alg:actexpand}8%\begin{algorithmic} % enter the algorithmic environment,%	\IF{$\compatablew{}{}\subseteq V_{\goal}$}%		\RETURN (P, h)%	\ENDIF(%	\STATE ${\rm AllActChoices}\gets \{\}$%	\FOR {$w\in\compatablew{}{}$}�%		\STATE ${\rm AllActChoices}\gets {\rm AllActChoices}\times w$.avblActs \textcolor{amethyst}{// Encode action choices for states in $\mathcal{W}$ cartesian product}	%	\ENDFOR2%	\FOR {${\rm actChoice} \in {\rm AllActChoices}$})%		\STATE ${\rm AllChosenActs}\gets \{\}$ %		\FOR {$w\in\compatablew{}{}$}q%			\STATE $A_w\gets {\rm actChoice}[w]$ \textcolor{amethyst}{// Obtain the set of action choices for each state}A%			\STATE ${\rm AllChosenActs}\gets {\rm AllChosenActs}\cup A_w$
%		\ENDFORm%		\STATE $P[\compatablew{}{}]\gets {\rm actChoice}$  \textcolor{amethyst}{// Put action choices in the plan}L%		\STATE ${\rm AllPartitions}\gets$ All partitions of ${\rm AllChosenActs}$2%		\FOR{${\rm partition} \in {\rm AllPartitions}$}*%			\STATE $(P', h')$ as a copy of $(P,h)$%			\STATE NoSoln$\gets\False$4%			\IF{$h'$ does not conflict with $\rm partition$}2%				\STATE $h'\gets h$.integrate($\rm partition$)/%				\FOR{\textit{group} $\in {\rm partition}$}`%					\STATE $\mathcal{W'}\gets$ vertices $\compatablew{}{}$ transitions to under \textit{group}i%					\IF{$\mathcal{W'}$ satisfies stipulation $\form{\Phi}$ $\land$ $P$.contains($\mathcal{W'}$)=\False}X%						\STATE $(P', h')\gets {\rm ObservationExpand}(\mathcal{W'}, P', h', \form{\Phi})$%						\IF{$(P',h')$ is empty}'%							\STATE ${\rm NoSoln}\gets\True$%							\STATE \textbf{break}
%						\ENDIF%					\ELSE&%						\STATE ${\rm NoSoln}\gets\True$%						\STATE \textbf{break}%					\ENDIF%				\ENDFOR"%				\IF{${\rm NoSoln}$ is \False}%					\RETURN $(P', h')$%				\ENDIF
%			\ENDIF
%		\ENDFOR	%	\ENDFOR%	\RETURN empty%\end{algorithmic}%\end{algorithm}%%\begin{algorithm}C%\caption{ObservationExpand($\compatablew{}{}, P, h, \form{\Phi}$)}%\label{alg:obsexpand}8%\begin{algorithmic} % enter the algorithmic environment,%	\IF{$\compatablew{}{}\subseteq V_{\goal}$}%		\RETURN (P, h)%	\ENDIF %	\STATE ${\rm AllObs}\gets\{\}$%	\FOR {$w\in\compatablew{}{}$}%		\STATE $O_w\gets w.$avblObs2%		\STATE ${\rm AllObs}\gets {\rm AllObs}\cup O_w$	%	\ENDFOR0%	\STATE $P[\compatablew{}{}]\gets {\rm AllObs}$D%	\STATE ${\rm AllPartitions}\gets$ All partitions of ${\rm AllObs}$1%	\FOR{${\rm partition} \in {\rm AllPartitions}$})%		\STATE $(P', h')$ as a copy of $(P,h)$#%		\STATE ${\rm NoSoln}\gets\False$5%		\IF{$h'$ does not conflict with ${\rm partition}$}.%			\FOR{\textit{group} $\in {\rm partition}$}_%				\STATE $\mathcal{W'}\gets$ vertices $\compatablew{}{}$ transitions to under \textit{group}h%				\IF{$\mathcal{W'}$ satisfies stipulation $\form{\Phi}$ $\land$ $P$.contains($\mathcal{W'}$)=\False}M%					\STATE $(P', h')\gets$ActionExpand($\mathcal{W'}, P', h', \form{\Phi}$)%					\IF{$(P',h')$ is empty}&%						\STATE ${\rm NoSoln}\gets\True$%						\STATE \textbf{break}%					\ENDIF
%				\ELSE%%					\STATE ${\rm NoSoln}\gets\True$%					\STATE \textbf{break}%				\ENDIF%			\ENDFOR	!%			\IF{${\rm NoSoln}$ is \False}%				\RETURN $(P', h')$
%			\ENDIF	%		\ENDIF	%	\ENDFOR%	\RETURN empty%\end{algorithmic}%\end{algorithm}MLet the number of actions and observations in $W$ be $|Y|$ and $|U|$, and thePnumber of vertices be $|V|$.  There are $2^{|U||V|}$ action choices to consider,Bin the worst case, for all the world states in $\compatablew{}{}$.@The total number of partitions is a Bell number $B_{|U|}$, whereP$B_{n+1}=\sum^{n}_{k=0} C^{k}_n B_k$ and $B_0=1$. For each partition, the numberMof groups we must consider is $|U|$. To expand an action vertex in the searchJtree, the computation complexity is $2^{|U||V|}|U|B_{|U|}$. Similarly, theOcomplexity to expand an observation vertex is $|Y|B_{|Y|}$. If the depth of theHtree is $d$, then the computational complexity is $O(2^{{|U||V|}^{d}})$.%, double exponential.\vspace*{-6pt}\section{Experimental results}\label{sec:experiments}\vspace*{-4pt}MWe implemented all the algorithms in this paper, the mainly using Python. TheTproblem \nameseekp\xspace was implemented with both the algorithm we propose and viaJspecification in computation tree logic (CTL) (and then utilizing the {\tt;nuXmv}  model-checker). All executions in this section used4a OSX laptop with a 2.4 GHz Intel Core i5 processor.LTo experiment we constructed a ${3\times 4}$ grid for the nuclear inspectionKscenario of \fref{fig:nuclear}.  Including the differing facility types andLradioactivity status, the world graph is a p-graph with $96$ vertices beforeOstate-determined expansion ($154$ vertices for the state-determined form).  TheKrobot can move left, right, up, down one block at a time. After the robot'sKmovement, it receives $5$ possible observations: pebble bed facility or notO(only when located at the blue star), radioactivity high or low when located atMone of the `{\bf ?}' cells, and cell is an exit.  But the observer only knowsLthe image of the actions and observations under a label map. The stipulationNrequires that the observer should learn the radioactivity strength, but shouldnever know the facility type.%\vspace*{-10pt}\begin{figure}
\centering%\hspace*{-15pt}"%\begin{subfigure}{0.41\linewidth}0% \includegraphics[scale=0.4]{figure/h-exp1.pdf}(%\caption{\label{fig:exp-nuclear-check}}% \end{subfigure}$\begin{subfigure}[c]{0.37\linewidth}/ \includegraphics[scale=0.4]{figure/h-exp2.pdf}+ \caption{\label{fig:exp-nuclear-findplan}}\end{subfigure}$\begin{subfigure}[c]{0.17\linewidth}/\includegraphics[scale=0.23]{figure/penta2.pdf}*\caption{\label{fig:exp-findplanlabelmap}}\end{subfigure}X\caption{The scenario and results for \nameseekp\xspace and \nameseekplm\xspace problem:N(a) shows the plan found in the nuclear inspection scenario, when the observerPknows nothing about robot's plan (The robot traces the gray arrow, then the blueKone if blue light is seen, the red one otherwise.) (b) shows the pentagonalBworld in \nameseekplm, where the robot moves along the gray lines.I%(a) shows the plan and label map to be checked in the nuclear inspectionP%scenario, when the observer knows nothing about robot's plan or the exact plan.N%(b) gives the plan found by \nameseekp\xspace with the given label map.  (ForP%(a) \& (b) plans can be understood as follows: the robot traces the gray arrow,O%then the blue one if blue light is seen, the red one otherwise.) (c) shows theN%pentagonal world in \nameseekplm, where the robot moves along the gray lines.\label{fig:exp-nuclear}}\end{figure}\vspace*{-20pt}E%Firstly, we {\sc check} whether the plan and label map pair given inN%\fref{fig:exp-nuclear-check} solve the problem. The plan reaches the goal butP%the observer cannot distinguish the radioactivity status when the full world isP%divulged (i.e., $D=W$). Also, it violates the stipulations because the facilityP%type is leaked when the exact plan is disclosed (i.e., $D=P$). Evaluation takes:%less than $1$ second in both tree-based and CTL methods. %Q%For this setting with the same label map and disclosed plan $D=W$, no satisfying=%solution exists, and hence \nameseekp\xspace returns \False.OFirstly, we {\sc seek} the plan in the nuclear inspection scenario with a labelKmap shown in \fref{fig:exp-nuclear-findplan}. A plan can be found (with theMworld graph disclosed, $D=W$). It takes $11$ seconds for the \aNd--\Or searchNand $24$ seconds for the CTL-based implementation to find their solutions. TheOCTL solver takes longer, but it prioritizes finding the plan of shortest lengthfirst.P%If there does not exists a plan with length $k$, then it searches the plan withC%length $k+1$ until the plan is found or maximum bound is reached. NThe plan found by CTL is shown in \fref{fig:exp-nuclear-findplan}. As the plan1found by \aNd--\Or search is lengthy, we omit it.G%Now altering the label map so that $h({\Uparrow})=h({\Downarrow})$ andN%$h({\Leftarrow})=h({\Rightarrow})$, a plan can be found (with the world graphK%disclosed, $D=W$). It takes $11$ seconds for the \aNd--\Or search and $24$Q%seconds for the CTL-based implementation to find their solutions. The CTL solverL%takes longer, but it prioritizes finding the plan of shortest length first.Q%%If there does not exists a plan with length $k$, then it searches the plan withD%%length $k+1$ until the plan is found or maximum bound is reached. O%The plan found by CTL is shown in \fref{fig:exp-nuclear-findplan}. As the plan2%found by \aNd--\Or search is lengthy, we omit it.NSince, for the nuclear inspection scenario, \nameseekplm\xspace doesn't returnPany result within reasonable time we opted to examine a smaller problem.  Here aMrobot moves in the pentagonal world shown in \fref{fig:exp-findplanlabelmap}.LThe robot can either decide to loop in the world ($a_1$) or exit the loop atOsome point ($a_2$ or $a_3$).  We wish to find a plan and label map pair so thatMthe robot can reach some charging station. The observer should not be able toGdistinguish the robot's position when at either of the top two charging
locations.I%The p-graph representation and the stipulation are shown in the right of!%\fref{fig:exp-findplanlabelmap}.P\nameseekplm\xspace gives a plan which moves forward $6$ times and then exits atPthe next time step. Additionally, to disguise the actions and observations afterHthe exit, it maps $h(a_2)=h(a_3)$ and $h(o_1)=h(o_3)$. Note that in thisKproblem, the robot reaches a goal, without considering the stipulations, byJtaking the exit at the next time step. The stipulations force the robot toMnavigate at least one loop in the world to conflate state for the sake of the	observer.>%Next, we will encode the search of label map into the search:%I%To search for a plan $(P, V_{\term})$ and label map $h$ for problem {\scN%Seek}$\big((W, V_{\goal}), ?, (I_f, ?), \text{?`}, \form{\Phi}\big)$, we will;%determine the actions to be taken at each world state fromK%$\mathcal{W}^{h\langle W\rangle, P}_{B}$, the label map $h$ and whether to&%terminate at $w$ if $w\in V_{\goal}$.%L%Firstly, if $w\in V_{\goal}$, we will determine whether the plan state $(w,;%\mathcal{W}^{h\langle W\rangle, P}_{B})\in V_{\term}$. LetO%$V=\mathcal{W}^{h\langle W\rangle, P}_{B}\cap V_{\goal}$. Depending on whetherL%to terminate at the vertices in $V$,  we will sample the choices encoded inM%power set $\mathcal{P}({V})$, where each $V'\in \mathcal{P}({V})$ representsO%the set of world states that are in $V_{\term}$.  Next, we are going to encode#%the label map into action choices:%\begin{lemma}%\label{lemma:act_lm_relation}L%Let the set of actions at $w$ be $L_w$. If there exists a solution for {\scK%Seek}$\big((W, V_{\goal}), ?, (I_f, ?), \text{?`}, \form{\Phi}\big)$, then>%there exists a solution which takes a set of actions $A$ fromP%$\mathcal{P}(L_w)$, i.e. the power set of $L_w$, and the actions in $L_w$ share%the same image.  \end{lemma}%\begin{proof}O%Let $(P, V_{\goal})$ and $h$ be a solution for {\sc Seek}$\big((W, V_{\goal}),L%?, (I_f, ?), \text{?`}, \form{\Phi}\big)$. We will construct a new solutionL%$(P', V_{\goal})$, which takes a set of actions $A\in \mathcal{P}(L_w)$ and%every %P%This lemma can be proved by contradiction. Suppose all actions in $L_w$ are notL%mapped to the same image under $h$. Then $h$ could partition $L_w$ into $m$P%groups $\{L_1, L_2,\dots, L_m\}$, where $m>1$. The observer is able to tell theL%difference between actions in $L_i$ and $L_j$, where $i\neq j$. Then we canO%construct a new plan $P_i$ with only $L_i$ taken at $(w,p)$, while the actionsH%taken at other vertices in $W\pgprod P$ stay the same. If $(P, h)$ is aM%solution, then $P_1$ is also a solution since the observer's I-states takingP%$P_1$ is a subset of those taking $P$. This contradicts with the condition thatM%no solution exists by removing at least one action action from $L_w$. Hence,H%actions in $L$ must be mapped to the same image under $h$.  \end{proof}N%Lemma~\ref{lemma:act_lm_relation} enables us to encode the label map into theJ%choice of actions. Let the set of available actions at world state $w$ beM%$L_w$. Then the power set $\mathcal{P}({L_w})\setminus \emptyset$ representsH%all the action-label map choice. Each $L\in \mathcal{P}({L_w})\setminusO%\emptyset$ represents that all actions in $L$ must be chosen and mapped to theK%same image. For example, if $L_w=\{a_1, a_2, a_3\}$, then $\{a_1, a_2\}\inP%\mathcal{P}({L_w})\setminus \emptyset$ represents that both $a_1$ and $a_2$ areN%chosen at this moment and $h(a_1)=h(a_2)$. Hence, we can reconstruct the planM%and label map from the choice of $\mathcal{P}({L_w})\setminus \emptyset$. InL%addition, lemma~\ref{lemma:act_lm_relation} tells us that if there exists aE%solution for {\sc Seek}$\big((W, V_{\goal}), ?, (I_f, ?), \text{?`},O%\form{\Phi}\big)$, then there exists a solution that only choose a choice fromO%$2^{L_w}$ when reaching world state $w$. Therefore, it is complete and correct1%to encode the label map into choices of actions.%5%For observation states, the plan should be ready to %%\begin{lemma}K%Let $(P,h)$ be a solution for {\sc Seek}$\big((W, V_{\goal}), ?, (I_f, ?),N%\text{?`}, \form{\Phi}\big)$, if there exists at least world states $a_1$ and%$a_2$ \end{lemma}O%According to Lemma~\ref{lemma:est_w_finest}, $\mathcal{W}^{h\langle W\rangle}$%\begin{lemma}H%If there exists a solution $(h,P)$ for the problem  {\sc Seek}$\big((W,I%V_{\goal}), ?, (I_f, ?), \text{?`}, \form{\Phi}\big)$, then there existsI%another solution $(h, P')$, where $P'$ is congruent on $W$.  \end{lemma}\vspace*{-8pt}\section{Conclusion}\label{sec:conclusion}\vspace*{-6pt}KThis paper continues a line of work on planning with constraints imposed onPknowledge- or belief-states. Our contribution is a substantial generalization ofOprior models, though, as we see in the section reporting experiments, with grimHimplications for computational requirements.  Future work might considerEtechniques that incorporate costs, informed methods (with appropriate?heuristics), and other ways to solve certain instances quickly.%\vspace*{-6pt}%\section*{Acknowledgements}%\vspace*{-6pt}%3%This work was supported by the NSF through awards I%\href{http://nsf.gov/awardsearch/showAward?AWD_ID=1453652}{IIS-1453652},I%\href{http://nsf.gov/awardsearch/showAward?AWD_ID=1527436}{IIS-1527436},%and I%\href{http://nsf.gov/awardsearch/showAward?AWD_ID=1526862}{IIS-1526862}.\vspace*{-6pt}\bibliographystyle{IEEEtran} \bibliography{mybib}\end{document}W%We will prove $\mathcal{W}^{h\langle W\rangle,D}_{h(s)}\subseteq \mathcal{W}^{I, D}_{h(s)}$ for all I-state graph $I$. For any $w\in \mathcal{W}^{h\langle W\rangle,D}_{h(s)}$, there exists an execution $s$ such that $B=\mathcal{V}^{h\langle W\rangle}_{h(s)}$ and $s\in \inv{h}[\mathbb{S}^{h\langle W\rangle}_{B}]\cap \Language{D}\cap \mathcal{S}^{W}_{w}$. Since $s\in \Language{D}\cap \mathcal{S}^{W}_{w}$, we should have $h(s)\in \Language{I}$. \textcolor{blue}{Otherwise, the observer may not be safe on all possible image execution.} Then we have $h(s)\in \mathbb{S}^{I}_{B'}$, where $B'=\mathcal{V}^{I}_{h(s)}$. Hence, $s\in \inv{h}[\mathbb{S}^{I}_{B'}]\cap \Language{D}\cap \mathcal{S}^{W}_{w}$. According to the definition, $w\in  \mathcal{W}^{I, D}_{h(s)}$. Therefore, $\mathcal{W}^{h\langle W\rangle,D}_{h(s)}\subseteq \mathcal{W}^{I, D}_{h(s)}$.5��